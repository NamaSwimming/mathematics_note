\documentclass[a4paper,11pt]{ltjsarticle}

% パッケージの読み込み
\usepackage{luatexja}
\usepackage{luatexja-fontspec}
\usepackage{auxhook}
\usepackage{graphicx}
\usepackage{adjustbox}
\usepackage{etoc} % etocパッケージを追加

% 日本語フォントの手動設定
\setmainjfont[
    YokoFeatures={JFM=jlreq},
    TateFeatures={JFM=jlreqv},
    BoldFont={Hiragino Kaku Gothic ProN W6}, % 太さを具体的に指定
]{Hiragino Mincho ProN}

\setsansjfont[
    YokoFeatures={JFM=jlreq},
    TateFeatures={JFM=jlreqv},
    BoldFont={Hiragino Kaku Gothic ProN W6}, % 太さを具体的に指定
]{Hiragino Kaku Gothic ProN}

% 英文フォントの設定
\setmainfont{Latin Modern Roman}[%
    BoldFont=lmroman10-bold,%
    BoldItalicFont=lmroman10-bolditalic,%
    ItalicFont=lmromanslant10-regular,%
    SmallCapsFont=lmromancaps10-regular,%
    SlantedFont=lmromanslant10-regular,%
    FontFace={sb}{n}{lmromandemi10-regular},%
    OpticalSize=0
]
\setsansfont{Latin Modern Sans}[%
    BoldFont=lmsans10-bold,%
    ItalicFont=lmsans10-oblique%
]
\setmonofont{Latin Modern Mono}[%
    BoldFont=lmmonolt10-bold,%
    ItalicFont=lmmono10-italic%
]

\renewcommand{\emph}[1]{\textbf{#1}}

% 数学パッケージの読み込み
\usepackage{amsmath}
\usepackage{amsfonts}
\usepackage{amssymb}
\usepackage{amsthm}
\usepackage{bm}
\usepackage{ascmac}

% ヘッダーフォントの設定
\renewcommand{\headfont}{\sffamily\bfseries}

% 括弧
\usepackage{delimseasy}

% 二段組
\usepackage{multicol}
\setlength{\columnseprule}{.5pt} % 中央の線

% 見出しのフォント
\renewcommand{\headfont}{\sffamily\bfseries}


%biblatex
\usepackage[style=numeric,sorting=none,hyperref=auto]{biblatex}
\DeclareNameAlias{default}{family-given}
\renewcommand*{\revsdnamepunct}{\addspace}
\DeclareDelimFormat{nametitledelim}{\addspace} % 名前とタイトルの間のデリミタをスペースに変更

%連名
\DeclareDelimFormat{multinamedelim}{\addcomma\space}
\DeclareDelimFormat{finalnamedelim}{\space and\space}
\DeclareDelimFormat{andothersdelim}{\space and\space}


\DeclareFieldFormat[book]{title}{\normalfont{#1}} % 本のタイトルを通常のフォントで表示
\addbibresource{bibliography.bib}

% 章の設定
\usepackage{titlesec}
\usepackage{picture}

% colors
\definecolor{myblue}{HTML}{3300CC}
\definecolor{myred}{HTML}{CC33CC}
\definecolor{mygreen}{HTML}{005500}
\definecolor{mynamecolor}{HTML}{5507FF}
\definecolor{myemailcolor}{HTML}{FF4F02}
\definecolor{darkbrown}{HTML}{880000}
\definecolor{dred}{HTML}{DD0000}
\definecolor{dgreen}{HTML}{005500}
\definecolor{dblue}{HTML}{000088}
\definecolor{mypurple}{HTML}{ba55d3} % 追加
\definecolor{burgundy}{rgb}{0.5, 0.0, 0.13}
\definecolor{charcoal}{rgb}{0.21, 0.27, 0.31}
\colorlet{lightmypurple}{mypurple!20!white}

% chapter
\titleformat{\part}[block]
{}{}{0pt}{
  \fontsize{16pt}{16pt}\normalfont\bfseries\filleft
}[
  \hrule \large{\filleft\textcolor{myred}{\thepart}} \\
  \vspace{-6mm}
]

% section
\titleformat{\section}[block]
{}{}{0pt}
{
  \colorbox{myred}{\begin{picture}(5,10)\end{picture}}
  \hspace{0pt}
  \normalfont \Large\bfseries 
  \hspace{-4pt}
}
[
\vspace{-4pt}
\begin{picture}(100,0)
  \linethickness{1pt}
  \put(3,14.4){\color{myred}\line(1,0){240}}
\end{picture}
\\
\vspace{-42pt}
]

% subsection
\titleformat{\subsection}[block]
{}{}{0pt}
{
  \colorbox{myblue}{\begin{picture}(4,8)\end{picture}}
  \hspace{0pt} 
  \normalfont \large\bfseries 
  \hspace{-4pt}
}
[
  \vspace{-1pt}
\begin{picture}(100,0)
  \linethickness{0.9pt}
  \put(3,13.2){\color{myblue}\line(1,0){216}}
\end{picture}
\\
\vspace{-36pt}
]

% 数式
\usepackage{nccmath}
\usepackage{mathtools}
\usepackage{empheq} % 数式の囲いに使う
\usepackage[bbsets]{jkmath} % \Nなどをつかえる
\usepackage{color}

% 箇条書き
\usepackage[shortlabels,inline]{enumitem}
\setlist[description]{font={\bfseries\sffamily}}

% まわりこみ
\usepackage{wrapfig}

% tcolorbox系
\usepackage[many]{tcolorbox}
\tcbuselibrary{breakable,skins,theorems,xparse}
\newtcolorbox{hosoibox}[1]{colframe=black,colback=white,coltitle=black,colbacktitle=white,boxrule=0.5pt,arc=0mm,enhanced,attach boxed title to top left={xshift=10mm,yshift=-3mm},boxed title style={frame hidden},title=#1}


%leftbar環境に注釈が入れられないことを解消する環境.名前は,tcolorboxの[t]とleftbarの組み合わせ
\newtcolorbox{tbleftline}{blanker,left=5mm,borderline west={1.1mm}{0pt}{black}}
\newenvironment{tleftbar}{\begin{tbleftline}\setlength{\parindent}{1\zw}}{\end{tbleftline}}


% 定理環境などの設定
% Theorem styles
\tcbset{mytheo/.style={fonttitle=\gtfamily\sffamily\bfseries\upshape,
enhanced,colframe=burgundy,colback=burgundy!2!white,colbacktitle=burgundy,
boxrule=0pt,borderline south={2pt}{-2pt}{burgundy},
left*=1\zw,right*=1\zw,
theorem style=standard,
breakable,sharp corners,
before skip=16pt,
after skip=18pt,
before upper={\setlength{\parindent}{1\zw}},
before lower={\setlength{\parindent}{1\zw}},
drop shadow={black!50!white}
}}

% Theorem
\newtcbtheorem[number within=subsection]{theorem}{Theorem}%
{mytheo}{th}
\newcommand{\thref}[1]{{\bfseries\sffamily Theorem \ref{th:#1}}}
% Proposition
\newtcbtheorem[use counter from=theorem]{prop}{Proposition}%
{mytheo}{pr}
\newcommand{\prref}[1]{{\bfseries\sffamily Proposition \ref{pr:#1}}}
% Corollary
\newtcbtheorem[use counter from=theorem]{cor}{Corollary}%
{mytheo}{co}
\newcommand{\coref}[1]{{\bfseries\sffamily Corollary \ref{co:#1}}}
% Axiom
\newtcbtheorem[use counter from=theorem]{axiom}{Axiom}%
{mytheo}{ax}
\newcommand{\axref}[1]{{\bfseries\sffamily Axiom \ref{ax:#1}}}
% Definition
\newtcbtheorem[use counter from=theorem]{definition}{Definition}%
{mytheo,
colframe=blue!50!black,colback=blue!50!black!2!white,colbacktitle=blue!50!black,borderline south={2pt}{-2pt}{blue!50!black},}{de}
\newcommand{\deref}[1]{{\bfseries\sffamily Definition \ref{de:#1}}}
% Lemma
\newtcbtheorem[use counter from=theorem]{lemma}{Lemma}%
{mytheo,
colframe=green!50!black,colback=green!50!black!2!white,colbacktitle=green!50!black,borderline south={2pt}{-2pt}{green!50!black},}{le}
\newcommand{\leqqref}[1]{{\bfseries\sffamily Lemma \ref{le:#1}}}
% Example
\newtcbtheorem[use counter from=theorem]{example}{Example}%
{mytheo,
colframe=myblue!50!black,colback=myblue!50!black!2!white,colbacktitle=myblue!50!black,,borderline south={2pt}{-2pt}{charcoal},}{ex}
\newcommand{\exref}[1]{{\bfseries\sffamily Example \ref{ex:#1}}}
% Question
\newtcbtheorem[use counter from=theorem]{question}{Question}%
{mytheo,
colframe=mypurple,colback=mypurple!2!white,colbacktitle=mypurple,borderline south={2pt}{-2pt}{mypurple},}{qu}
\newcommand{\quref}[1]{{\bfseries\sffamily Question \ref{qu:#1}}}

% その他の設定
% 囲い枠
\DeclareTColorBox{simplesquarebox}{ o m O{.5} O{} }% 
    {empty, left=2mm, right=2mm, top=-1mm, attach boxed title to top left={xshift=1.2\zw},
    boxed title style={empty,left=-2mm,right=-2mm}, colframe=black, coltitle=black, coltext=black, breakable,  
    underlay unbroken={\draw[black,line width=#3pt]
        (title.east) -- (title.east-|frame.east) -- (frame.south east) -- (frame.south west) -- (title.west-|frame.west) -- (title.west); },
    underlay first={\draw[black,line width=#3pt](title.east) -- (title.east-|frame.east) -- (frame.south east) ;
        \draw[black,line width=#3pt] (frame.south west) -- (title.west-|frame.west) -- (title.west); },
    underlay middle={\draw[black,line width=#3pt](frame.north east) -- (frame.south east) ;
        \draw[black,line width=#3pt](frame.south west) -- (frame.north west) ;},
    underlay last={\draw[black,line width=#3pt](frame.north east) -- (frame.south east) -- (frame.south west) -- (frame.north west) ;},
    fonttitle=\gtfamily, IfValueTF={#1}{title=【#2】〈#1〉}{title=【#2】},#4}

% 例題環境
\newcounter{mondaibangou}
\newtcolorbox{mondai}[1][]{enhanced,boxrule=0.5mm,
        top=2pt,left=44pt,right=4pt,bottom=2pt,arc=0mm,
        colframe=blue!30!gray,
        boxrule=1pt,
        underlay={
        \node[inner sep=1pt,blue!50!black,fill=blue!10!white]at ([xshift=22pt,yshift=-9pt]interior.north west) {\stepcounter{mondaibangou}\bfseries\gtfamily 問題\themondaibangou};},
        segmentation code={%
        \draw[dashed] (segmentation.west)--(segmentation.east);
        \node[inner sep=1pt,blue!50!black,fill=blue!10!white] at ([xshift=22pt,yshift=-8pt]segmentation.south west) {\bfseries\gtfamily 解答};},
        before upper={\setlength{\parindent}{1\zw}},
        before lower={\setlength{\parindent}{1\zw}},#1
}

% 感想環境
\DeclareTColorBox{kans}{ o m O{3} O{}}%
{enhanced, colback=white, colframe=white,
attach boxed title to top left={xshift=1cm,yshift=-\tcboxedtitleheight/2}, fonttitle=\bfseries,varwidth boxed title=0.85\linewidth, coltitle=black, fonttitle=\gtfamily, 
enlarge top by=2mm, enlarge bottom by=2mm, breakable, sharp corners,
boxed title style={colback=white,left=0mm,right=0mm}, 
borderline={.75pt}{#3pt}{black,dotted},
% underlay settings...
IfValueTF={#1}{title=【#2】〈#1〉}{title=【#2】},#4}

\usepackage{framed,color}

% 題名付き四角
\usepackage{ascmac}
\usepackage{fancybox}

% 図に使うもの
\usepackage{tikz}
\usetikzlibrary{intersections,calc,arrows.meta,calendar,shadows.blur}
\usepackage{tikz-3dplot}
\usepackage[marginparwidth=0pt,margin=20truemm]{geometry}
\usepackage{bxpapersize}
\usepackage[absolute,overlay]{textpos} % 図の配置を好きにする

\usepackage{pgfplots}
\usepgfplotslibrary{fillbetween}
\pgfplotsset{compat=1.17}

% footnoteの変更
\renewcommand\thefootnote{{\dag}\arabic{footnote}}
\renewcommand{\thempfootnote}{{\dag}\arabic{mpfootnote}}
\interfootnotelinepenalty=10000

\usepackage{oubraces} % overunderbraces

% underbraceの文字数が多いときのためのadunderbrace
\usepackage{ifthen}
\newlength{\wdTempA}
\newlength{\wdTempB}
\newcommand{\adunderbrace}[2]{%
\settowidth{\wdTempA}{$#1$}%
\settowidth{\wdTempB}{${\scriptstyle #2}$}%
\ifthenelse{\wdTempA<\wdTempB}{%
\hspace*{.5\wdTempA}\hspace*{-.5\wdTempB}%
\underbrace{#1}_{#2}%
\hspace*{.5\wdTempA}\hspace*{-.5\wdTempB}%
}{%
\underbrace{#1}_{#2}%
}%
}%
% 丸付き文字
\newcommand{\ctext}[1]{\raise0.2ex\hbox{\textcircled{\scriptsize{#1}}}}

\setlength{\abovedisplayskip}{5pt} 
\setlength{\belowdisplayskip}{3pt}
% ユーザー定義
\newcommand{\dash}[1]{#1^\prime}
\newcommand{\ddash}[1]{#1^{\prime\prime}}
\newcommand{\dddash}[1]{#1^{\prime\prime\prime}}
\newcommand{\hodash}[2]{#2^{(#1)}}
\renewcommand{\labelenumi}{(\arabic{enumi})}% itemを(数字)に変更
\newcommand{\two}{I\hspace{-1.2pt}I}
\newcommand{\three}{I\hspace{-1.2pt}I\hspace{-1.2pt}I}
\DeclareMathOperator{\Ker}{Ker}
\renewcommand{\proofname}{\textgt{証明}}

\renewcommand{\le}{\leqq}
\renewcommand{\ge}{\geqq}

\renewcommand{\leq}{\leqq}
\renewcommand{\geq}{\geqq}

\newcommand\kakko[1]{\noindent\textbf{《#1》}}

% ハイパーリンク用
\usepackage{url}
\usepackage{hyperref}
\usepackage{xcolor}
\hypersetup{colorlinks=true,citecolor={mypurple!40!black},linkcolor={mypurple!40!black},urlcolor={mypurple!70!black},}
\usepackage{bookmark}

\everymath{\displaystyle}

\newcommand{\HRule}[1]{\rule{\linewidth}{#1}}

\AtBeginDocument{\RenewCommandCopy\qty\SI}

% 丸付き文字
\usepackage{docmute}

\everymath{\displaystyle}

\etocsetnexttocdepth{subsection} % 必要に応じて調整

\etocsettocstyle{
\vskip10pt
  \noindent
  \begin{tikzpicture}[x=\textwidth]
    % 全幅の線を描く
    \draw[line width=2pt, blur shadow,color=mypurple] (0,0) -- (1,0);
    % 「目次」のボックスを線の中心に配置
    \node[
      draw=mypurple,
      line width=3pt,
      fill=white,
      inner sep=6pt,
      blur shadow,
      rounded corners
    ] at (0.5,0) {\bfseries\Large\textcolor{mypurple}{目次}};
  \end{tikzpicture}
  \vskip10pt
}{
  \vskip10pt
  \noindent
  \begin{tikzpicture}[x=\textwidth]
    \draw[line width=2pt,blur shadow, color=mypurple] (0,0) -- (1,0);
  \end{tikzpicture}
  \vskip20pt
}
\makeatother

% 各レベルの目次項目のスタイルを設定
% `\leftskip` を統一します
\etocsetstyle{part}
{\noindent}
{%
  \vskip20pt
  \parindent0pt \leftskip0pt % `part` はインデントなし
  \etoclink{\bfseries\huge\color{mypurple}\etocname}\par
  \vskip10pt
}
{}


\etocsetstyle{section}
{\noindent}
{%
  \vskip10pt
  \parindent0pt \leftskip0pt % `\leftskip` を part と同じにします
  \etoclink{\bfseries\Large\color{mypurple}\etocname}%
  \leaders\hbox to0.5em{\hss.\hss}\hfill
  \etocpage\par
  \vskip10pt
}
{}

\etocsetstyle{subsection}
{\noindent}
{%
  \parindent0pt \leftskip0pt % `\leftskip` を section と同じにします
  \etoclink{\bfseries\color{mypurple}\etocname}%
  \space % 項目の間に半角スペースを挿入
  (\etocnumber,~p.\etocpage)% 章番号とページ番号の表示
  \hspace{1em}
}
{}

% タイトルセクションをロード
\usepackage{titletoc}

% タイトルページを定義
\usepackage{pagecolor} % 背景色を変更するためのパッケージ

\newcommand{\tituloum}[5]{
  \begin{titlepage}
    \begin{center}
        \pagecolor{white} % 背景色を白に設定
        \color{mypurple} % テキストカラーをmypurpleに設定
        
        \vspace*{2\baselineskip}
        
        % 横線の色をmypurpleに設定
        \textcolor{mypurple}{\rule{\textwidth}{1.6pt}}\vspace*{-\baselineskip}\vspace*{2pt}
        \textcolor{mypurple}{\rule{\textwidth}{0.4pt}}
        
        \vspace{0.75\baselineskip}
        
        {\huge #1}
        
        \vspace{0.75\baselineskip}
        
        \textcolor{mypurple}{\rule{\textwidth}{0.4pt}}\vspace*{-\baselineskip}\vspace{3.2pt}
        \textcolor{mypurple}{\rule{\textwidth}{1.6pt}}
        
        \vspace{2\baselineskip}
        
        #3
        
        \vspace*{3\baselineskip}
        
        {\huge #2}
        
        \vspace{0.5\baselineskip}
        
        \textit{#4}
        
        \vfill
        
        \vspace{0.3\baselineskip}
        
        #5
        
    \end{center}
\end{titlepage}%
}
\begin{document}

% タイトルページの呼び出し
\tituloum{数学のノート}{なまちゃん}{}{}{\today}

%\include{Preambulo/Capa}

\thispagestyle{empty}

\newpage
\pagenumbering{arabic}
\pagecolor{white}

% 目次の表示
\localtableofcontents
\newpage 

\part*{数学の基本}

\phantomsection
\section{用語}

\phantomsection
\subsection{はじめに}
大学数学を学ぶにあたって,基本的な用語や記法に慣れておくことは重要である.そのことを踏まえた上で,高校で学んだ用語や記法を復習していこう.
まず,以下の用語についてみていくことにする:
\begin{itemize}
	\item 定義
	\item 命題
	\item 定理
	\item 補題・系
	\item 証明
\end{itemize}
\phantomsection
\subsection{定義}

\emph{定義(definition)}とは,用語の意味を明確に述べたものであり,\textbf{Def}と略記される.同じ事柄について,2つ以上の定義の形式があることもある.次の例を見てみよう:

\begin{example}{「絶対値」の定義}{「絶対値」の定義}
	$ x \in \mathbb{R}$に対して,$x$の絶対値を$\abs{x}$とかき,次のように定義する:
	\[
	\abs{x} \coloneqq \max \{ x , -x \} .
	\]
	この定義は以下のような形式で表現してもよい:
	\[
	\abs{x} \coloneqq  \sqrt{x^2}.
	\]
\end{example}

次に,定義の性質についてみていこう.
線型代数の講義で学ぶことであるが,逆行列の定義は以下のようになる:

\begin{example}{「逆行列」の定義}{「逆行列」の定義}
	正方行列$A$に対して,$BA = AB =E$となるような$B$が存在するとき,このような$B$を$A$の逆行列という.
\end{example}


ここで注意するのは \emph{数学では定義は最小限の情報にとどめることが慣習となっている}ということである.たとえば,\exref{「逆行列」の定義}の定義から,以下の命題が成り立ち,その証明は容易である.

\begin{prop}{逆行列の一意性}{逆行列の一意性}
	正方行列$A$に対して,$A$の逆行列は存在するとすればただひとつである.
\end{prop}

\begin{tleftbar}
	\begin{proof}
	\prref{逆行列の一意性}の証明は,$A$の逆行列が$B$,$C$であるとすると,
	\[
		AB=BA=E,\quad AC=CA=E
	\]
	が成り立つ.このとき,
	\[
		B=BE=B(AC)=(BA)C=EC=C
	\]
	が成り立つ.よって,$B=C$であり,逆行列の一意性が示された.
	\end{proof}
\end{tleftbar}

このことから,\exref{「逆行列」の定義}でとりあげた逆行列の定義をもっと詳しく

\begin{quote}
	正方行列$A$に対して,$BA = AB =E$となるような$B$が存在するとき,このような$B$はただひとつで,$B$を$A$の逆行列という.
	$B$は一意に存在するので,これを$A^{-1}$と記す\footnote{一意に存在することがわかっていないと,$A^{-1}$のような記法で表すことはためらわれる.}.もちろん
	\[
		AA^{-1}=A^{-1}A=E.
	\]
\end{quote}
と,「逆行列の一意性を定義に含めてもいいのではないか.」と主張する人がいるかもしれない.
ただ,数学では「定義は最小限の情報にとどめ,そこから導かれる主張を命題として証明する」という慣習があり,
なにが「定義」で,なにが「証明すべきこと」であるかはっきりさせることが多い\footnote{ただ,実際はここで取り上げた逆行列の定義も情報過多である.線型代数で学ぶことになるが,$AB=E$と$BA=E$のどちらか片方の式のみで定義していいからである.}.

\phantomsection
\subsection{命題}


\emph{命題(proposition)}とは,真偽が定まっている文を指す\footnote{命題はふたつの意味があり,ここでいう命題は「真偽が決まっている文」というニュアンスを持つ「広い意味での命題」というよりかは「定理・命題・補題」などと並列して表記される「狭い意味での命題」である.「狭い意味での命題」は正しい主張である.
	たとえば,「$3$以上の自然数$n$に対して$x^n + y^n =z^n$は自然数解を持たない」という「フェルマーの最終定理」は,
	アンドリュー・ワイルズが証明するまでは「真偽が決まっているがどちらかはわからない」という「広い意味での命題」であったが,
	証明されたのちに「広い意味での命題」であると同時に「狭い意味での命題」にもなった.}.\textbf{Prop}と略記される.
このような論理体系を\emph{二値論理}という.

中間として扱われる主張には
\begin{enumerate}[(1)]
	\item 定義が曖昧なもの \label{enu:定義が曖昧なもの}
	\item 意味が曖昧なもの \label{enu:意味が曖昧なもの}
	\item パラドックス \label{enu:パラドックス}
\end{enumerate}
などがある.\ref{enu:定義が曖昧なもの},\ref{enu:意味が曖昧なもの}についてはのちほど説明するとして,ここでは\ref{enu:パラドックス}について例をあげよう.

\begin{example}{自己言及のパラドックス}{自己言及のパラドックス}
	次のような主張を考える:
	\begin{quote}
		この文は偽である.
	\end{quote}
	この主張は,自己言及のパラドックスであり,真偽が決まらない.
	
	以下も自己言及のパラドックスの例である:
	\begin{quote}
		「この壁に貼り紙をしてはならない」と書かれた貼り紙
	\end{quote}
\end{example}

ここまでで,二値論理でとりあげない主張を述べてきたが,そろそろ二値論理で取り扱う主張のお話に戻ろう.
また,以下では簡単のために,「広い意味での命題」と「狭い意味での命題」のどちらも「命題」と記すと約束する.


\begin{example}{命題}{命題}
    次に示す文は真偽が真の命題である.
    \begin{enumerate}[(1)]
        \item 「1+1=2」
        \item 「円周率は100未満である」
        \item 「信号の色は3色である」
        \item 「霞ヶ浦は日本で二番目に面積が大きい湖である」
    \end{enumerate}
    また,次に示す文は真偽が偽の命題である.
    \begin{enumerate}[(a)]
        \item 「1+1=46」
        \item 「円周率は3未満である」
        \item 「信号の色は5色である」
        \item 「霞ヶ浦は日本で五番目に面積が大きい湖である」
    \end{enumerate}
\end{example}

\begin{example}{命題でない文}{命題でない文}
    次に示す文は命題ではない
    \begin{enumerate}[(A)]
    \item 「$1+1$」(なにも主張しておらず,真か偽か判定できない)
    \item 「霞ヶ浦の面積は大きい」(客観的に大きいか判定できない)
    \item 「桃はおいしい」(基準が明確でなく,客観的に真か偽か判定できない)
    \item 「$x^2 > 4$」($x$に具体的な値を代入しないと真か偽か判定できない)
    \end{enumerate}
\end{example}

\phantomsection
\subsection{定理}

\emph{定理(theorem)}とは,正しいと分かっている数学の主張\footnote{つまり,「狭い意味での命題」のこと.}の中でもとりわけ重要なものを指す.\textbf{Thm}と略記される.
定理は重要な主張であり,「余弦定理」,「加法定理」,「ハイネ・ボレルの被覆定理」など,固有の名前が与えられているものも存在する.

\begin{example}{三平方の定理}{三平方の定理}
	三平方の定理の主張は以下のようになる:
	\begin{quotation}
    直角三角形の斜辺の長さを$c$とし,他の二辺の長さを$a$,$b$としたとき,
    \[
        a^2+b^2=c^2
    \]
    が成り立つ.
\end{quotation}
\end{example}

\begin{example}{次元定理}{次元定理}
	次元定理の主張は以下のようになる:
	\begin{quotation}
	$A$を$m \times n$実行列とするとき,
	\[
	\rank A + \dim (\ker A) =n .
	\]
	\end{quotation}
\end{example}

\phantomsection
\subsection{補題・系}

\emph{補題(lemma)}とは,定理や命題を示す過程で補助的に使われる命題のことである.\textbf{Lem}と略記される.
補題は命題の一種である.

\emph{系(corollary)}とは,先に示した定理の主張からただちに得られる命題のことである.\textbf{Cor}と略記される.

補題,系を説明するにあたっては,証明すべき事柄が複数個必要である.その一例をここに示す.


\begin{example}{補題}{補題}
	ユークリッドの補題の主張は以下の通りで,素因数分解の一意性を示すために用いられる:
	\begin{quote}
	素数$p$が$ab$を割り切るなら,$p$は$a$と$b$の少なくとも一方を割り切る.
	\end{quote}
\end{example}

そして素因数分解の一意性から,次の命題が導かれる.

\begin{example}{自然数の約数の個数}{自然数の約数の個数}
	自然数$n$の約数の個数$d(n)$は以下の通りで,素因数分解の一意性からただちにわかる.
	\begin{quote}
	\[
	d(n)= (e_1+1)(e_2+1)\dots(e_i+1)
	\]
	ただし$n = p_1^{e_1} p_2^{e_2} \dots p_i^{e_1}$とする.
	\end{quote}
\end{example}

このことから,以下の系が導かれることは明らかであろう.

\begin{example}{系}{系}
	自然数$n$の約数の個数$d(n)$が奇数となる必要十分条件は,$n$が完全平方数であることである.
\end{example}

\phantomsection
\subsection{証明}

ここまでで,正しいか裏付けられた主張を「命題」,「定理」,「補題・系」などとよぶことをみてきた.
これらの主張が正しいことを示すプロセスを\emph{証明(proof)}という.

証明は一つの命題にいくつか存在する場合がほとんどであり,たとえば,\exref{三平方の定理}の証明は100通り以上も存在することが知られている.

\begin{example}{証明}{証明}
辺の長さが$a$と$b$の直角三角形を$4$つ用意する.
これらの三角形を組み合わせて,辺の長さが$a + b$の正方形を作る.
このとき,大きな正方形の面積は$(a + b)^2$である.

一方で,大きな正方形は中央に辺の長さが$c$の小さな正方形と、$4$つの三角形で構成されている.

よって,大きな正方形の面積は,小さい正方形の面積と4つの三角形の面積の和に等しい:
\[
(a + b)^2 = c^2 + 4 \left( \frac{1}{2}ab \right).
\]
この式を整理すると,
\[
 (a + b)^2 = c^2 + 2ab
\]
であるから,
\begin{align*}
& a^2 + 2ab + b^2 = c^2 + 2ab, \\
& \therefore ~a^2 + b^2 = c^2.
\end{align*}
これが証明すべきことであった.\qedhere 
\begin{center}
\begin{tikzpicture}[scale=0.75]
% 辺の長さの定義
\def\a{3}
\def\b{4}
\def\c{5}

% 大きな正方形の描画
\draw (0,0) rectangle (\a+\b,\a+\b);

% 4つの直角三角形の描画
%第1の三角形
\draw (0,0) -- (\b,0) -- (0,\a) -- cycle;
\draw (0,\a) -- node[midway,above] {$c$} (\b,0);

% 第2の三角形
\draw (\b,0) -- (\a+\b,0) -- (\a+\b,\b) -- cycle;
\draw (\a+\b,\b) -- node[midway,above] {$c$} (\b,0);

% 第3の三角形
\draw (\a+\b,\b) -- (\a+\b,\a+\b) -- (\a,\a+\b) -- cycle;
\draw (\a,\a+\b) -- node[midway,above] {$c$} (\a+\b,\b);

% 第4の三角形
\draw (\a, \a+\b) -- (0, \a+\b) -- (0,\a) -- cycle;
\draw (0,\a) -- node[midway,above] {$c$} (\a, \a+\b);

% 中央の小さな正方形の描画
\draw (\b,0) -- (\a+\b,\b) -- (\a,\a+\b) -- (0,\a) -- cycle;

% 寸法のラベル
\draw [<->] (-0.5,0) -- (-0.5,\a) node[midway,left] {$a$};
\draw [<->] (-0.5,\a) -- (-0.5,\a+\b) node[midway,left] {$b$};
\draw [<->] (0,-0.5) -- (\b,-0.5) node[midway,below] {$b$};
\draw [<->] (\b,-0.5) -- (\a+\b,-0.5) node[midway,below] {$a$};

\end{tikzpicture}
\end{center}
\end{example}


\phantomsection
\section{記法}
\phantomsection
\subsection{論理記号}

	論理記号は,命題を結合するために用いられる記号である.代表的な論理記号を列挙してみよう:
	\begin{table}[htbp]
		\centering
		\caption{論理記号とその意味}
		\begin{tabular}{c|c}
		\hline
		記号 & 意味 \\
		\hline
		$\land$ & かつ(論理積) \\
		$\lor$ & または(論理和) \\
		$\lnot$ & 否定 \\
		$\to$ & ならば \\
		$\Leftrightarrow$ & 必要十分条件 \\
		\hline
		\end{tabular}
		\end{table}

    これらの論理記号を用いて,命題を結合することができる.たとえば,次のような命題を考えてみよう:

\begin{example}{論理記号}{論理記号}
  $\mathrm{P}$を「$x$は偶数である」とし,$\mathrm{Q}$を「$x$は$3$の倍数である」とする.

  \begin{center}
  \begin{tabular}{c|l}
  \hline
  記号 & 意味 \\
  \hline \hline
  $\mathrm{P} \land \mathrm{Q}$ & $x$は偶数であり,かつ$3$の倍数である \\
  $\mathrm{P} \lor \mathrm{Q}$ &$x$は偶数である,または$3$の倍数である \\
  $\lnot \mathrm{P}$ & $x$は偶数でない \\
  \hline
  \end{tabular}
\end{center}
\end{example}

	\phantomsection
\subsection{集合}

\begin{definition}{集合}{集合}
    もののあつまりを\emph{集合}という\footnote{公理的集合論の立場では,集合とは「無定義語」であるが,ここで詳しくは触れない.}集合を構成する物を\emph{元}または\emph{要素}といい,集合$A$の元が$a$であることを$a \in A$,$A \ni a$などと表す.
\end{definition}

集合に関して,いくつか注意点を挙げよう.

\begin{itemize}
    \item 集合では書き並べる順序が重要でないため,例えば $\{1, 2, 3\} = \{3, 2, 1\}$である.
    \item 同じ要素が重複して含まれていても,1つの要素として扱われるため,例えば$\{1, 1, 2, 2, 2, 3\} = \{1, 2, 3\}$である.
    \item 集合の要素には,種類が異なるものを同時に含めることができる.例えば,$\{4, \{3\}\}$では,$4$は数であり,$\{3\}$は集合であるが,集合としての資格がある.
\end{itemize}


\begin{definition}{部分集合}{部分集合}
	集合$A$の要素はすべて集合$B$の要素でもあるとき,$A$は$B$の\emph{部分集合}であるといい,これを
	\[
	A \subset B,\quad B \supset A
	\]
	などと表す.
\end{definition}

\begin{shadebox}
	よく使う集合
    \begin{itemize}
    \item $\mathbb{R}$は実数全体の集合を表している.Real numberの頭文字をとった.
    \item $\mathbb{N}=\{0,1,2,\cdots \}$は自然数全体の集合を表している.Natural number の頭文字をとった.
    \item $\mathbb{C}=\{ a+bi \mid a,b \in \mathbb{R}\}$は複素数全体の集合を表している.Complex number の頭文字をとった.
    \item $\mathbb{Z}=\{0, \pm 1 , \pm 2 , \cdots\}$は整数全体の集合を表している.Zはドイツ語由来である,
    \item $\mathbb{Q}=\{ a/b \mid   a,b \in \mathbb{Z},b \ne 0 \}$は有理数全体の集合を表している.「商」を表すイタリア語由来である.
    \end{itemize}
\end{shadebox}

\begin{example}{}{}
    \[
		x \in \mathbb{Q}
	\]
	と書くことで,$x$は有理数であることを表す.
\end{example}

\begin{example}{}{}
    \[
		\mathbb{R} \subset \mathbb{C}
	\]
	である.つまり,実数全体の集合は複素数全体の集合の部分集合である.
\end{example}

集合の表記は文脈により省略されることがある.たとえば,以下のような問題があったとする.
\begin{quotation}
	2次方程式
	\[
	x^2 - 3x -4 =0
	\]
	を解け.
\end{quotation}
$x$が属する全体集合は定められていないが,この場合だと「$x \in \mathbb{C}$」とされることが多い.
よってこの方程式の解は$x = -1 , 4$とする場合が多い.
だが,もちろん$ x \in \mathbb{N}$とするなら,$ -1 \notin \mathbb{N}$なので,この場合の解は$ x= 4$のみである.
ただ,$x$が属する全体集合は,文脈でわかったり明記されている場合が多いので,あまり心配はいらないと筆者は考える.
\phantomsection
\section{言い回し}
\phantomsection
\subsection{存在}



数学の証明において,「存在」は重要な概念である.といっても,あまりこのことを意識したことのない読者の方もいると思うので,この場を借りて具体例をもとに説明を試みることとする.

まず、「最大値・最小値の定理」を考えてみる.この定理は,
\begin{quotation}
	$[a,b]$で連続な関数$f$に対して,$f$は$[a,b]$上で最大値と最小値を持つ.
\end{quotation}
というものある.

「最大値・最小値が存在するなんて当たり前だ」と思われる読者もいるかもしれない.しかし,本当にそれは自明なのか.
実際には,関数の連続性や区間の閉有界性といった条件が揃って初めて,最大値や最小値の「存在」を保証できる.この定理を証明するにあたっては,厳密な数学的議論が必要となる.

次に,指数関数を考える際に有理数列$(x_n)_{n \in \mathbb{N}}$を用いる場合の例を見てみよう:

$x \in \mathbb{R}$としたとき,$a^x$を定義するために,
\[
	\lim_{n \to \infty} a^{x_n} = a^x
\]

とするが,この極限が存在することは本当に自明なのか.
$a^2$や$a^{2/3}$の具体的な値のイメージは思い浮かぶと思うが,たとえば$a^{\pi}$のイメージについてはどうであろうか.

このように,極限の存在を証明するためには,有理数列の収束性など,細かな数学的性質を確認する必要がある.このようにして初めて,指数関数の定義が厳密なものとなるのである.

\subsection{一意性}

数学において「一意性」が重要である場面は多い.読者の中には線型代数の講義で「逆行列の一意性」などに触れた方もいると思われる.なぜ重要であるのか,一つ例を挙げて考えてることとする.

微分積分の講義で習う定理に「平均値の定理」というものがある.
その主張は
\begin{quotation}
	$[a,b]$で連続,$(a,b)$で微分可能な関数$f$に対して,
	\[
	\frac{f(b)-f(a)}{b-a}=f'(c)
	\]
	をみたす$ c \in (a,b)$が存在する.
\end{quotation}

というものである.証明はのちに述べるとして,この定理の主張を少し変更してみよう:

\begin{quotation}
	$[a,b]$で連続,$(a,b)$で微分可能な関数$f$に対して,
	\[
	\frac{f(b)-f(a)}{b-a}=f'(c)
	\]
	をみたす$ c \in (a,b)$がただひとつ存在する.
\end{quotation}

ここでは「$c \in (a,b)$が存在する」という主張を「$c \in (a,b)$がただひとつ存在する」というより強い主張に変更した.この主張の真偽は偽である.以下で,このことが問題になる状況を挙げよう.

\begin{wrapfigure}[11]{r}{0.55\textwidth}
	\centering
\begin{tikzpicture}
    \begin{axis}[
        axis lines=middle,
        xmin=-1.2, xmax=1.2,
        ymin=-1.5, ymax=1.5,
        xtick=\empty, ytick=\empty, % 目盛を消す
        xlabel={}, ylabel={},       % ラベルを消す
        samples=200,
        domain=-1.2:1.2,
        width=10cm,
        height=8cm,
    ]
    % 関数 f(x) = x^3 のグラフ
    \addplot [dblue, thick] {x^3};

    % 点 (-1, -1) と (1, 1) を結ぶ割線
    \addplot [dred, thick] coordinates {(-1, -1) (1, 1)}; 

    % 平均値の定理を満たす点 x = -1/√3 の接線
    \def\xone{-1/sqrt(3)}
    \def\yone{(\xone)^3}
    \addplot [dgreen, dashed] {1*(x - \xone) + \yone};

    % 平均値の定理を満たす点 x = 1/√3 の接線
    \def\xtwo{1/sqrt(3)}
    \def\ytwo{(\xtwo)^3}
    \addplot [dgreen, dashed] {1*(x - \xtwo) + \ytwo};

    % 重要な点をマーク
    \addplot [only marks, mark=*, mark options={fill=black}] coordinates {(\xone, \yone) (\xtwo, \ytwo)};
    \addplot [only marks, mark=*, mark options={fill=dred}] coordinates {(-1, -1) (1, 1)}; % 赤い丸に変更
    \end{axis}
\end{tikzpicture}
\end{wrapfigure}
$f(x)=x^3$という関数を考える.この関数は$[-1,1]$で連続,$(-1,1)$で微分可能である.

このとき,図のように,条件を満たす$c \in (-1,1)$は複数存在するので,「ただひとつ存在する」という主張は偽である.
さらに言えば,2本より多くこのような接線を引ける場合もある.

このことから,安易に「ただ一つ存在する」などと強い主張をすることは避けるべきであることがわかる.このことは平均値の定理に限らず,中間値の定理なども同様である.


\phantomsection
\subsection{かつ/または}



数学と日常における「または」の使い方は異なる.以下に例を挙げよう.

\begin{example}{「または」の使用例}{「または」の使用例}
    \begin{enumerate}[(A)]
	\item ランチメニューの主食として,米またはパンがついてくる\footnote{この文を「米とパンの両方が食べられる」と解釈してもらっては困る.}.\label{enu:米またはパン}
	\item 「運転免許を持っていない人」または「18歳未満の人」はレンタカーを借りることができない. \label{enu:運転免許または18歳未満}
	\end{enumerate}
\ref{enu:米またはパン}は「どちらか片方のみ」の意味で「または」を使い,\ref{enu:運転免許または18歳未満}は「いずれかが」の意味で「または」を用いている.
\end{example}

数学では,「または」は「いずれかが」の意味で使われ,「どちらか片方のみ」の意味で使われることはない.

たとえば,$A$,$B$を集合とするとき,
\[
x \in A \cup B
\]
は「$x$は$A$の元であるか,$B$の元であるか,あるいは両方である」という意味である.
つまり,「$ x \in A$であり,$x \notin  B$である」あるいは「$ x \notin A$であり,$x \in B$である」といった状況のときにも,$x \in A \cup B$と記す.

\newpage 


\cite{nakajima},\cite{kaneko}を参考にした.

\printbibliography[title=参考文献]

\end{document}