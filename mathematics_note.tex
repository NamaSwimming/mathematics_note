\documentclass[a4paper,11pt]{ltjsarticle}

% パッケージの読み込み
\usepackage{luatexja}
\usepackage{luatexja-fontspec}
\usepackage{auxhook}
\usepackage{graphicx}
\usepackage{adjustbox}
\usepackage{etoc} % etocパッケージを追加

% 日本語フォントの手動設定
\setmainjfont[
    YokoFeatures={JFM=jlreq},
    TateFeatures={JFM=jlreqv},
    BoldFont={Hiragino Kaku Gothic ProN W6}, % 太さを具体的に指定
]{Hiragino Mincho ProN}

\setsansjfont[
    YokoFeatures={JFM=jlreq},
    TateFeatures={JFM=jlreqv},
    BoldFont={Hiragino Kaku Gothic ProN W6}, % 太さを具体的に指定
]{Hiragino Kaku Gothic ProN}

% 英文フォントの設定
\setmainfont{Latin Modern Roman}[%
    BoldFont=lmroman10-bold,%
    BoldItalicFont=lmroman10-bolditalic,%
    ItalicFont=lmromanslant10-regular,%
    SmallCapsFont=lmromancaps10-regular,%
    SlantedFont=lmromanslant10-regular,%
    FontFace={sb}{n}{lmromandemi10-regular},%
    OpticalSize=0
]
\setsansfont{Latin Modern Sans}[%
    BoldFont=lmsans10-bold,%
    ItalicFont=lmsans10-oblique%
]
\setmonofont{Latin Modern Mono}[%
    BoldFont=lmmonolt10-bold,%
    ItalicFont=lmmono10-italic%
]

\renewcommand{\emph}[1]{\textbf{#1}}

% 数学パッケージの読み込み
\usepackage{amsmath}
\usepackage{amsfonts}
\usepackage{amssymb}
\usepackage{amsthm}
\usepackage{bm}
\usepackage{ascmac}

% ヘッダーフォントの設定
\renewcommand{\headfont}{\sffamily\bfseries}

% 括弧
\usepackage{delimseasy}

% 二段組
\usepackage{multicol}
\setlength{\columnseprule}{.5pt} % 中央の線

% 見出しのフォント
\renewcommand{\headfont}{\sffamily\bfseries}


%biblatex
\usepackage[style=numeric,sorting=none,hyperref=auto]{biblatex}
\DeclareNameAlias{default}{family-given}
\renewcommand*{\revsdnamepunct}{\addspace}
\DeclareDelimFormat{nametitledelim}{\addspace} % 名前とタイトルの間のデリミタをスペースに変更

%連名
\DeclareDelimFormat{multinamedelim}{\addcomma\space}
\DeclareDelimFormat{finalnamedelim}{\space and\space}
\DeclareDelimFormat{andothersdelim}{\space and\space}


\DeclareFieldFormat[book]{title}{\normalfont{#1}} % 本のタイトルを通常のフォントで表示
\addbibresource{bibliography.bib}

% 章の設定
\usepackage{titlesec}
\usepackage{picture}

% colors
\definecolor{myblue}{HTML}{3300CC}
\definecolor{myred}{HTML}{CC33CC}
\definecolor{mypink}{HTML}{FF66FF}
\definecolor{mygreen}{HTML}{005500}
\definecolor{mynamecolor}{HTML}{5507FF}
\definecolor{myemailcolor}{HTML}{FF4F02}
\definecolor{darkbrown}{HTML}{880000}
\definecolor{dred}{HTML}{DD0000}
\definecolor{dgreen}{HTML}{005500}
\definecolor{dblue}{HTML}{000088}
\definecolor{mypurple}{HTML}{ba55d3} % 追加
\definecolor{burgundy}{rgb}{0.5, 0.0, 0.13}
\definecolor{charcoal}{rgb}{0.21, 0.27, 0.31}
\colorlet{lightmypurple}{mypurple!20!white}

% chapter (part) with number
\titleformat{\part}[block]
{}{}{0pt}{
  \fontsize{16pt}{16pt}\normalfont\bfseries\filleft \thepart\quad
}[
  \hrule \large{\filleft\textcolor{myred}{\thepart}} \\
  \vspace{-6mm}
]

% section with number
\titleformat{\section}[block]
{}{}{0pt}
{
  \colorbox{myred}{\begin{picture}(5,10)\end{picture}}
  \hspace{0pt}
  \normalfont \Large\bfseries 
  \hspace{-4pt} \thesection\quad
}
[
\vspace{-4pt}
\begin{picture}(100,0)
  \linethickness{1pt}
  \put(3,14.4){\color{myred}\line(1,0){240}}
\end{picture}
\\
\vspace{-46pt}
]

% subsection with number
\titleformat{\subsection}[block]
{}{}{0pt}
{
  \colorbox{myblue}{\begin{picture}(4,8)\end{picture}}
  \hspace{0pt} 
  \normalfont \large\bfseries 
  \hspace{-4pt} \thesubsection\quad
}
[
  \vspace{-1pt}
\begin{picture}(100,0)
  \linethickness{0.9pt}
  \put(3,13.2){\color{myblue}\line(1,0){216}}
\end{picture}
\\
\vspace{-42pt}
]

% subsubsection with number
\titleformat{\subsubsection}[block]
{}{}{0pt}
{
  \colorbox{charcoal}{\begin{picture}(3,6)\end{picture}}
  \hspace{0pt} 
  \normalfont \bfseries 
  \hspace{-4pt} \thesubsubsection\quad
}
[
  \vspace{-1pt}
\begin{picture}(100,0)
  \linethickness{0.8pt}
  \put(3,12.2){\color{charcoal}\line(1,0){160}}
\end{picture}
\\
\vspace{-28pt}
]
% 数式
\usepackage{nccmath}
\usepackage{mathtools}
\usepackage{empheq} % 数式の囲いに使う
\usepackage[bbsets]{jkmath} % \Nなどをつかえる
\usepackage{color}

% 箇条書き
\usepackage[shortlabels,inline]{enumitem}
\setlist[description]{font={\bfseries\sffamily}}

% まわりこみ
\usepackage{wrapfig}

% tcolorbox系
\usepackage[many]{tcolorbox}
\tcbuselibrary{breakable,skins,theorems,xparse}
\newtcolorbox{hosoibox}[1]{colframe=black,colback=white,coltitle=black,colbacktitle=white,boxrule=0.5pt,arc=0mm,enhanced,attach boxed title to top left={xshift=10mm,yshift=-3mm},boxed title style={frame hidden},title=#1}


%leftbar環境に注釈が入れられないことを解消する環境.名前は,tcolorboxの[t]とleftbarの組み合わせ
\newtcolorbox{tbleftline}{blanker,left=5mm,borderline west={1.1mm}{0pt}{black}}
\newenvironment{tleftbar}{\begin{tbleftline}\setlength{\parindent}{1\zw}}{\end{tbleftline}}


% 定理環境などの設定
% Theorem styles
\tcbset{mytheo/.style={fonttitle=\gtfamily\sffamily\bfseries\upshape,
enhanced,colframe=burgundy,colback=burgundy!2!white,colbacktitle=burgundy,
boxrule=0pt,borderline south={2pt}{-2pt}{burgundy},
left*=1\zw,right*=1\zw,
theorem style=standard,
breakable,sharp corners,
before skip=16pt,
after skip=18pt,
before upper={\setlength{\parindent}{1\zw}},
before lower={\setlength{\parindent}{1\zw}},
drop shadow={black!50!white}
}}

% Theorem
\newtcbtheorem[number within=subsection]{theorem}{Theorem}%
{mytheo}{th}
\newcommand{\thref}[1]{{\bfseries\sffamily Theorem \ref{th:#1}}}
% Proposition
\newtcbtheorem[use counter from=theorem]{prop}{Proposition}%
{mytheo}{pr}
\newcommand{\prref}[1]{{\bfseries\sffamily Proposition \ref{pr:#1}}}
% Corollary
\newtcbtheorem[use counter from=theorem]{cor}{Corollary}%
{mytheo}{co}
\newcommand{\coref}[1]{{\bfseries\sffamily Corollary \ref{co:#1}}}
% Axiom
\newtcbtheorem[use counter from=theorem]{axiom}{Axiom}%
{mytheo}{ax}
\newcommand{\axref}[1]{{\bfseries\sffamily Axiom \ref{ax:#1}}}
% Definition
\newtcbtheorem[use counter from=theorem]{definition}{Definition}%
{mytheo,
colframe=blue!50!black,colback=blue!50!black!2!white,colbacktitle=blue!50!black,borderline south={2pt}{-2pt}{blue!50!black},}{de}
\newcommand{\deref}[1]{{\bfseries\sffamily Definition \ref{de:#1}}}
% Lemma
\newtcbtheorem[use counter from=theorem]{lemma}{Lemma}%
{mytheo,
colframe=green!50!black,colback=green!50!black!2!white,colbacktitle=green!50!black,borderline south={2pt}{-2pt}{green!50!black},}{le}
\newcommand{\leqqref}[1]{{\bfseries\sffamily Lemma \ref{le:#1}}}
% Example
\newtcbtheorem[use counter from=theorem]{example}{Example}%
{mytheo,
colframe=mypink!50!black,colback=mypink!50!black!2!white,colbacktitle=mypink!50!black,,borderline south={2pt}{-2pt}{mypink!50!black},}{ex}
\newcommand{\exref}[1]{{\bfseries\sffamily Example \ref{ex:#1}}}
% Question
\newtcbtheorem[use counter from=theorem]{question}{Question}%
{mytheo,
colframe=mypurple,colback=mypurple!2!white,colbacktitle=mypurple,borderline south={2pt}{-2pt}{mypurple},}{qu}
\newcommand{\quref}[1]{{\bfseries\sffamily Question \ref{qu:#1}}}

% その他の設定
% 囲い枠
\DeclareTColorBox{simplesquarebox}{ o m O{.5} O{} }% 
    {empty, left=2mm, right=2mm, top=-1mm, attach boxed title to top left={xshift=1.2\zw},
    boxed title style={empty,left=-2mm,right=-2mm}, colframe=black, coltitle=black, coltext=black, breakable,  
    underlay unbroken={\draw[black,line width=#3pt]
        (title.east) -- (title.east-|frame.east) -- (frame.south east) -- (frame.south west) -- (title.west-|frame.west) -- (title.west); },
    underlay first={\draw[black,line width=#3pt](title.east) -- (title.east-|frame.east) -- (frame.south east) ;
        \draw[black,line width=#3pt] (frame.south west) -- (title.west-|frame.west) -- (title.west); },
    underlay middle={\draw[black,line width=#3pt](frame.north east) -- (frame.south east) ;
        \draw[black,line width=#3pt](frame.south west) -- (frame.north west) ;},
    underlay last={\draw[black,line width=#3pt](frame.north east) -- (frame.south east) -- (frame.south west) -- (frame.north west) ;},
    fonttitle=\gtfamily, IfValueTF={#1}{title=【#2】〈#1〉}{title=【#2】},#4}

% 例題環境
\newcounter{mondaibangou}
\newtcolorbox{mondai}[1][]{enhanced,boxrule=0.5mm,
        top=2pt,left=44pt,right=4pt,bottom=2pt,arc=0mm,
        colframe=blue!30!gray,
        boxrule=1pt,
        underlay={
        \node[inner sep=1pt,blue!50!black,fill=blue!10!white]at ([xshift=22pt,yshift=-9pt]interior.north west) {\stepcounter{mondaibangou}\bfseries\gtfamily 問題\themondaibangou};},
        segmentation code={%
        \draw[dashed] (segmentation.west)--(segmentation.east);
        \node[inner sep=1pt,blue!50!black,fill=blue!10!white] at ([xshift=22pt,yshift=-8pt]segmentation.south west) {\bfseries\gtfamily 解答};},
        before upper={\setlength{\parindent}{1\zw}},
        before lower={\setlength{\parindent}{1\zw}},#1
}

% 感想環境
\DeclareTColorBox{kans}{ o m O{3} O{}}%
{enhanced, colback=white, colframe=white,
attach boxed title to top left={xshift=1cm,yshift=-\tcboxedtitleheight/2}, fonttitle=\bfseries,varwidth boxed title=0.85\linewidth, coltitle=black, fonttitle=\gtfamily, 
enlarge top by=2mm, enlarge bottom by=2mm, breakable, sharp corners,
boxed title style={colback=white,left=0mm,right=0mm}, 
borderline={.75pt}{#3pt}{black,dotted},
% underlay settings...
IfValueTF={#1}{title=【#2】〈#1〉}{title=【#2】},#4}

\usepackage{framed,color}

% 題名付き四角
\usepackage{ascmac}
\usepackage{fancybox}

% 図に使うもの
\usepackage{tikz}
\usetikzlibrary{intersections,calc,arrows.meta,calendar,shadows.blur}
\usepackage{tikz-3dplot}
\usepackage[marginparwidth=0pt,margin=20truemm]{geometry}
\usepackage{bxpapersize}
\usepackage[absolute,overlay]{textpos} % 図の配置を好きにする

\usepackage{pgfplots}
\usepgfplotslibrary{fillbetween}
\pgfplotsset{compat=1.17}

% footnoteの変更
\renewcommand\thefootnote{{\dag}\arabic{footnote}}
\renewcommand{\thempfootnote}{{\dag}\arabic{mpfootnote}}
\interfootnotelinepenalty=10000

\usepackage{oubraces} % overunderbraces

% underbraceの文字数が多いときのためのadunderbrace
\usepackage{ifthen}
\newlength{\wdTempA}
\newlength{\wdTempB}
\newcommand{\adunderbrace}[2]{%
\settowidth{\wdTempA}{$#1$}%
\settowidth{\wdTempB}{${\scriptstyle #2}$}%
\ifthenelse{\wdTempA<\wdTempB}{%
\hspace*{.5\wdTempA}\hspace*{-.5\wdTempB}%
\underbrace{#1}_{#2}%
\hspace*{.5\wdTempA}\hspace*{-.5\wdTempB}%
}{%
\underbrace{#1}_{#2}%
}%
}%
% 丸付き文字
\newcommand{\ctext}[1]{\raise0.2ex\hbox{\textcircled{\scriptsize{#1}}}}

\setlength{\abovedisplayskip}{5pt} 
\setlength{\belowdisplayskip}{3pt}
% ユーザー定義
\newcommand{\dash}[1]{#1^\prime}
\newcommand{\ddash}[1]{#1^{\prime\prime}}
\newcommand{\dddash}[1]{#1^{\prime\prime\prime}}
\newcommand{\hodash}[2]{#2^{(#1)}}
\renewcommand{\labelenumi}{(\arabic{enumi})}% itemを(数字)に変更
\newcommand{\two}{I\hspace{-1.2pt}I}
\newcommand{\three}{I\hspace{-1.2pt}I\hspace{-1.2pt}I}
\DeclareMathOperator{\Ker}{Ker}
\renewcommand{\proofname}{\textgt{証明}}

\renewcommand{\le}{\leqq}
\renewcommand{\ge}{\geqq}

\renewcommand{\leq}{\leqq}
\renewcommand{\geq}{\geqq}

\newcommand\kakko[1]{\noindent\textbf{《#1》}}

% ハイパーリンク用
\usepackage{url}
\usepackage{hyperref}
\usepackage{xcolor}
\hypersetup{colorlinks=true,citecolor={mypurple!40!black},linkcolor={mypurple!40!black},urlcolor={mypurple!70!black},}
\usepackage{bookmark}

\everymath{\displaystyle}

\newcommand{\HRule}[1]{\rule{\linewidth}{#1}}

\AtBeginDocument{\RenewCommandCopy\qty\SI}

% 丸付き文字
\usepackage{docmute}

\everymath{\displaystyle}

\etocsetnexttocdepth{subsubsection} % 必要に応じて調整

  \etocsettocstyle{
\vskip10pt
  \noindent\hspace*{-\leftskip}%
  \begin{tikzpicture}[x=\linewidth]
    % 全幅の線を描く
    \draw[line width=2pt, blur shadow,color=mypurple] (0,0) -- (1,0);
    % 「目次」のボックスを線の中心に配置
    \node[
      draw=mypurple,
      line width=3pt,
      fill=white,
      inner sep=6pt,
      blur shadow,
      rounded corners
    ] at (0.5,0) {\bfseries\Large\textcolor{mypurple}{目次}};
  \end{tikzpicture}
  \vskip10pt
}{
  \vskip10pt
  \noindent\hspace*{-\leftskip}%
  \begin{tikzpicture}[x=\linewidth]
    \draw[line width=2pt, blur shadow, color=mypurple] (0,0) -- (1,0);
  \end{tikzpicture}
  \vskip20pt
}
\makeatother

% 各レベルの目次項目のスタイルを設定
% `\leftskip` を階層に応じて設定します
\etocsetstyle{part}
{\noindent}
{%
  \vskip20pt
  \parindent0pt \leftskip0pt % `part` はインデントなし
  \etoclink{\bfseries\huge\color{mypurple}\etocname}\par
  \vskip10pt
}
{}

\etocsetstyle{section}
{\noindent}
{%
  \vskip10pt
  \parindent0pt \leftskip5pt % `section` は10ptのインデント
  \etoclink{\bfseries\LARGE\color{mypurple}\etocnumber~\etocname}%
  \leaders\hbox to 0.5em{\hss.\hss}\hfill
  \etocpage\par
  \vskip10pt
}
{}

\etocsetstyle{subsection}
{\noindent}
{%
  \vskip6pt
  \parindent0pt \leftskip25pt % `subsection` は20ptのインデント
  \etoclink{\bfseries\large\color{mypurple}\etocnumber~\etocname}%
  \leaders\hbox to 0.5em{\hss.\hss}\hfill
  \etocpage\par
  \vskip6pt
}
{}

\etocsetstyle{subsubsection}
{\noindent}
{%
  \parindent0pt \leftskip50pt % `subsubsection` は30ptのインデント
  \etoclink{\bfseries\normalsize\color{mypurple}\etocnumber~\etocname}%
  \space % 項目の間に半角スペースを挿入
  (p.\etocpage)% ページ番号の表示
  \hspace{1em}
}
{}

% タイトルセクションをロード
\usepackage{titletoc}

% タイトルページを定義
\usepackage{pagecolor} % 背景色を変更するためのパッケージ

\newcommand{\tituloum}[5]{
  \begin{titlepage}
    \begin{center}
        \pagecolor{white} % 背景色を白に設定
        \color{mypurple} % テキストカラーをmypurpleに設定
        
        \vspace*{2\baselineskip}
        
        % 横線の色をmypurpleに設定
        \textcolor{mypurple}{\rule{\textwidth}{1.6pt}}\vspace*{-\baselineskip}\vspace*{2pt}
        \textcolor{mypurple}{\rule{\textwidth}{0.4pt}}
        
        \vspace{0.75\baselineskip}
        
        {\huge #1}
        
        \vspace{0.75\baselineskip}
        
        \textcolor{mypurple}{\rule{\textwidth}{0.4pt}}\vspace*{-\baselineskip}\vspace{3.2pt}
        \textcolor{mypurple}{\rule{\textwidth}{1.6pt}}
        
        \vspace{2\baselineskip}
        
        #3
        
        \vspace*{3\baselineskip}
        
        {\huge #2}
        
        \vspace{0.5\baselineskip}
        
        \textit{#4}
        
        \vfill
        
        \vspace{0.3\baselineskip}
        
        #5
        
    \end{center}
\end{titlepage}%
}

\setcounter{secnumdepth}{3}

\begin{document}

% タイトルページの呼び出し
\tituloum{数学のノート}{なまちゃん}{}{}{\today}

%\include{Preambulo/Capa}

\thispagestyle{empty}

\newpage
\pagenumbering{arabic}
\pagecolor{white}

% 目次の表示
\localtableofcontents
\newpage 

\part{数学の基本}
\phantomsection
\section{集合}
  \phantomsection
  \subsection{集合の定義と表記法}

  \subsubsection{集合の定義}

  現代数学の基礎をなす概念に「集合」や「写像」がある.まず「集合」からみていこう.
  \begin{definition}{集合}{集合}
      もののあつまりを\emph{集合}という\footnote{公理的集合論の立場では,集合とは「無定義語」であるが,ここで詳しくは触れない.}.集合を構成する物を\emph{元}または\emph{要素}といい,集合$A$の元が$a$であることを$a \in A$,$A \ni a$などと表す.
  \end{definition}
  
  集合に関して,いくつか注意点を挙げよう.
  
  \begin{itemize}
      \item 集合では書き並べる順序が重要でないため,例えば $\{1, 2, 3\} = \{3, 2, 1\}$である.
      \item 同じ要素が重複して含まれていても,1つの要素として扱われるため,例えば$\{1, 1, 2, 2, 2, 3\} = \{1, 2, 3\}$である.
      \item 集合の要素には,種類が異なるものを同時に含めることができる.例えば,$\{4, \{3\}\}$では,$4$は数であり,$\{3\}$は集合であるが,集合としての資格がある.
  \end{itemize}

  「ある集合の要素を部分的に含んでいる集合」を考えることは,数学において重要な意義を持つ.例えば,ある集合の特定の性質を持つ要素のみを集めた部分集合を考えることで,問題解決や証明において焦点を絞ることができる.
  といっても,現段階で部分集合のイメージを掴むことは難しいので,まず定義を確認して,部分集合の具体的な例はのちほど紹介することにする.
  
  \begin{definition}{部分集合}{部分集合}
    集合$A$の要素はすべて集合$B$の要素でもあるとき,$A$は$B$の\emph{部分集合}であるといい,これを
    \[
    A \subset B,\quad B \supset A
    \]
    などと表す\footnote{このことを$ A \subseteq B$,$B \supseteq A$とかくこともある.一般に,$A \subset B$,$ B \supset A$と書いた場合には$A=B$の場合を含んで意味をとる.}.
  \end{definition}


  \subsubsection{集合の表記}
  
  数学を勉強する上で,よく使う集合には固有の記号を与える場合が多い.以下ではよく使う集合の例を挙げよう.

  \begin{shadebox}
    よく使う集合
      \begin{itemize}
      \item $\mathbb{R}$は実数全体の集合を表している.Real numberの頭文字をとった.
      \item $\mathbb{N}=\{0,1,2,\cdots \}$は自然数全体の集合を表している.Natural number の頭文字をとった.
      \item $\mathbb{C}=\{ a+bi \mid a,b \in \mathbb{R}\}$は複素数全体の集合を表している.Complex number の頭文字をとった.
      \item $\mathbb{Z}=\{0, \pm 1 , \pm 2 , \cdots\}$は整数全体の集合を表している.Zはドイツ語由来である,
      \item $\mathbb{Q}=\{ a/b \mid   a,b \in \mathbb{Z},b \ne 0 \}$は有理数全体の集合を表している.「商」を表すイタリア語由来である.
      \end{itemize}
  \end{shadebox}


\begin{shadebox}
  数を表す集合以外にも,固有の表記が与えられている集合が存在する.
  \begin{itemize}
  \item $\varnothing$ \footnote{空集合は$\emptyset$と表記することもある.ギリシャ文字の$\phi$で代用されることもあるが,本来の空集合の記号はノルウェー語由来である.}は要素を一つも持たない集合,すなわち\emph{空集合}を表している.
  \item $\mathcal{P}(A)$は集合$A$のすべての部分集合からなる集合,すなわち\emph{べき集合}を表している.
 \end{itemize}
\end{shadebox}
  
  \begin{example}{}{}
      \[
      x \in \mathbb{Q}
    \]
    と書くことで,$x$は有理数であることを表す.
  \end{example}
  
  \begin{example}{}{}
      \[
      \mathbb{R} \subset \mathbb{C}
    \]
    である.つまり,実数全体の集合は複素数全体の集合の部分集合である.
  \end{example}


  \begin{example}{}{}
    $A$を集合とするとき,
    \[
    \varnothing  \subset A
  \]
  である.つまり,空集合は全ての集合の部分集合である.
\end{example}

  \begin{example}{べき集合}{べき集合}
    $A=\{ 1, 2\}$とすると,
    \[
    \mathcal{P}(A) = \{ \varnothing, \{1\}, \{2\}, \{1, 2\}\}
  \]
  である.
\end{example}
集合の表記は文脈により省略されることがある.たとえば,以下のような問題があったとする.
\begin{quotation}
  2次方程式
  \[
  x^2 - 3x -4 =0
  \]
  を解け.
\end{quotation}
$x$が属する全体集合は定められていないが,この場合だと「$x \in \mathbb{C}$」とされることが多い.
よってこの方程式の解は$x = -1 , 4$とする場合が多い.
だが,もちろん$ x \in \mathbb{N}$とするなら,$ -1 \notin \mathbb{N}$なので,この場合の解は$ x= 4$のみである.
ただ,$x$が属する全体集合は,文脈でわかったり明記されている場合が多いので,あまり心配はいらないと筆者は考える.


\begin{prop}{集合の分配律}{集合の分配律}
  $A$,$B$,$C$を集合とするとき,
  \begin{align*} 
    A \cap ( B \cup C) &= (A \cap B) \cup (A \cap C), \\
    A \cup ( B \cap C) &= (A \cup B) \cap (A \cup C).
  \end{align*}
\end{prop}

\begin{tleftbar}
\begin{proof}
$ x \in A \cap ( B \cup C)$とすると,$x \in A$かつ$x \in B \cup C$である.$x \in B \cup C$とすると,$x \in B$または$x \in C$である.
よって,$x \in A$かつ$x \in B$であるとき,$x \in A \cap B$である.また,$x \in A$かつ$x \in C$であるとき,$x \in A \cap C$である.
これらのことから,$x \in A \cap ( B \cup C)$とすると,$x \in (A \cap B) \cup (A \cap C)$である.
つまり,
\[
A \cap ( B \cup C) \subset (A \cap B) \cup (A \cap C).
\]
逆に,$x \in (A \cap B) \cup (A \cap C)$とすると,$x \in A \cap B$または$x \in A \cap C$である.
$x \in A \cap B$とすると,$x \in A$かつ$x \in B$である.また,$x \in A \cap C$とすると,$x \in A$かつ$x \in C$である.
よって,$x \in A$かつ$x \in B$または$x \in A$かつ$x \in C$である.これらのことから$x \in A$かつ$x \in B \cup C$である.
つまり,
\[
(A \cap B) \cup (A \cap C) \subset A \cap ( B \cup C).
\]
以上の考察により,$A \cap ( B \cup C) = (A \cap B) \cup (A \cap C)$である.

\end{proof}
\end{tleftbar}

  \phantomsection
  \subsection{集合の演算}
  
  集合に対して,以下のような演算を定義することができる.
  
  \begin{definition}{共通部分・和集合・差集合・補集合}{共通部分・和集合・差集合・補集合}
      $A$と$B$を集合とするとき,以下の定義をする.
      \begin{description}[labelwidth=4em]
        \item[共通部分] $A$と$B$の両方に属する要素全体の集合を$A \cap B$と表す.
        \item[和集合] $A$または$B$のいずれかに属する要素全体の集合を$A \cup B$と表す.
        \item[差集合]$A$の要素で$B$に属さないもの全体の集合を$A \setminus B$と表す.
        \item[補集合] 全体集合$X$に対して,$A$の補集合を$A^c = X \setminus A$と表す.
      \end{description}
  \end{definition}
  
  \begin{example}{集合}{集合}
      例えば,$A = \{1, 2, 3\}$,$B = \{3, 4, 5\}$とすると,
      \begin{itemize*}
          \item $A \cap B = \{3\}$
          \item $A \cup B = \{1, 2, 3, 4, 5\}$
          \item $A \setminus B = \{1, 2\}$
          \item $X=\{1,2,3,4,5\}$とすると,$A^c = \{4, 5\}$
      \end{itemize*}
  \end{example}
  
  \begin{figure}[h]
    \centering
    \begin{minipage}{0.45\linewidth}
      \centering
      % A∩B のベン図
      \begin{tikzpicture}
        % 全体集合の長方形
        \draw (0,0) rectangle (6,4) node[below left] {$X$};
        % 集合Aの円
        \draw (2,2) circle (1.5) node[left] {$A$};
        % 集合Bの円
        \draw (4,2) circle (1.5) node[right] {$B$};
        % 共通部分を塗りつぶす
        \begin{scope}
          \clip (2,2) circle (1.5);
          \fill[gray] (4,2) circle (1.5);
        \end{scope}
      \end{tikzpicture}
      \caption{$A \cap B$ のベン図}
    \end{minipage}
    \hfill
    \begin{minipage}{0.45\linewidth}
      \centering
      % A∪B のベン図
      \begin{tikzpicture}
        % 全体集合の長方形
        \draw (0,0) rectangle (6,4) node[below left] {$X$};
        % 集合Aの円を塗りつぶす
        \fill[gray] (2,2) circle (1.5);
        % 集合Bの円を塗りつぶす
        \fill[gray] (4,2) circle (1.5);
        % 集合Aの円の輪郭
        \draw (2,2) circle (1.5) node[left] {$A$};
        % 集合Bの円の輪郭
        \draw (4,2) circle (1.5) node[right] {$B$};
      \end{tikzpicture}
      \caption{$A \cup B$ のベン図}
    \end{minipage}

    \vspace{2mm}

    \begin{minipage}{0.45\linewidth}
      \centering
      % A - B のベン図
      \begin{tikzpicture}
        % 全体集合の長方形
        \draw (0,0) rectangle (6,4) node[below left] {$X$};
        % 集合Aの円
        \draw (2,2) circle (1.5) node[left] {$A$};
        % 集合Bの円
        \draw (4,2) circle (1.5) node[right] {$B$};
        % A - B を塗りつぶす
        \begin{scope}
          \clip (2,2) circle (1.5);
          \fill[gray] (0,0) rectangle (6,4);
        \end{scope}
        \begin{scope}
          \clip (4,2) circle (1.5);
          \fill[white] (0,0) rectangle (6,4);
        \end{scope}
        % 集合Aの円の輪郭
        \draw (2,2) circle (1.5) node[left] {$A$};
      \end{tikzpicture}
      \caption{$A \setminus B$ のベン図}
    \end{minipage}
    \hfill
    \begin{minipage}{0.45\linewidth}
      \centering
      % A^c のベン図
      \begin{tikzpicture}
        % 全体集合の長方形を塗りつぶす
        \fill[gray] (0,0) rectangle (6,4);
        % 集合Aの円を白で塗りつぶす
        \fill[white] (2,2) circle (1.5);
        % 全体集合の枠
        \draw (0,0) rectangle (6,4) node[below left] {$X$};
        % 集合Aの円の輪郭
        \draw (2,2) circle (1.5) node[left] {$A$};
      \end{tikzpicture}
      \caption{$A^c$ のベン図}
    \end{minipage}
  \end{figure}

  集合はしばしば条件を用いて記述される.
  
  \begin{description}[labelwidth=4em]
      \item[列挙形式] $\{ a, b, c \}$のように要素を列挙して表す.
      \item[条件形式] $\{ x \in \mathbb{N} \mid x \leq 5 \}$のように条件を用いて表す.
  \end{description}

  \begin{example}{集合の表記}{集合の表記}
    \[
    \{ 1,2,3 \} = \{ n \in \mathbb{N} \mid  n^2 -4n +3 \leqq 0 \}
    \]
  \end{example}
\phantomsection
\section{論理と証明}
\subsection{論理}
\subsubsection{論理記号}

	論理記号は,命題を結合するために用いられる記号である.代表的な論理記号を列挙してみよう:
	\begin{table}[htbp]
		\centering
		\caption{論理記号とその意味}
		\begin{tabular}{c|c}
		\hline
		記号 & 意味 \\
		\hline
    $\land$ & かつ(論理積) \\
    $\lor$ & または(論理和) \\
    $\lnot$ & 否定 \\
    $\to$ & ならば(含意) \\
    $\Leftrightarrow$ & 必要十分条件(同値) \\
    $\forall$ & 任意の(全称量化) \\
    $\exists$ & 存在する(存在量化) \\
		\hline
		\end{tabular}
		\end{table}

    これらの論理記号を用いて,命題を結合し,複雑な論理式を構成することができる.特に,量化記号である$\forall$と$\exists$を用いることで,数学的な主張を一般的に表現することができる.

    \subsubsection{量化記号}
    
    
    \begin{definition}{全称命題}{全称命題}
        \[
          \forall\, x \, : \, P(x)
        \]
        は,「すべての$x$について,$P(x)$が成り立つ」という意味であり,これを\emph{全称命題}という.
    \end{definition}
    

    \begin{definition}{存在命題}{存在命題}
        \[
        \exists \,  x\, : \, P(x)
        \]
        は,「$P(x)$が成り立つような$x$が存在する」という意味であり,これを\emph{存在命題}という.
    \end{definition}
    
    \begin{example}{量化記号の例}{量化記号の例}
        \begin{itemize}
            \item $\forall x \in \mathbb{R}\, : \, x^2 \geq 0$\\
            「任意の実数$x$について,$x^2$は$0$以上である.」
            \item $\exists x \in \mathbb{Z} \, : \, x^2 = 4$\\
            「ある整数$x$が存在して,$x^2 = 4$となる.」
        \end{itemize}
    \end{example}

    \subsection{証明}

    \subsubsection{証明とは}
    数学において,主張が正しいことを示すプロセスを\emph{証明(proof)}という.
    
    証明は一つの命題にいくつか存在する場合がほとんどであり,たとえば,\exref{三平方の定理}の証明は100通り以上も存在することが知られている.
    
    \begin{example}{証明}{証明}
    辺の長さが$a$と$b$の直角三角形を$4$つ用意する.
    これらの三角形を組み合わせて,辺の長さが$a + b$の正方形を作る.
    このとき,大きな正方形の面積は$(a + b)^2$である.
    
    一方で,大きな正方形は中央に辺の長さが$c$の小さな正方形と,$4$つの三角形で構成されている.
    
    よって,大きな正方形の面積は,小さい正方形の面積と4つの三角形の面積の和に等しい:
    \[
    (a + b)^2 = c^2 + 4 \left( \frac{1}{2}ab \right).
    \]
    この式を整理すると,
    \[
     (a + b)^2 = c^2 + 2ab
    \]
    であるから,
    \begin{align*}
    & a^2 + 2ab + b^2 = c^2 + 2ab, \\
    & \therefore ~a^2 + b^2 = c^2.
    \end{align*}
    これが証明すべきことであった.\qedhere 
    \begin{center}
    \begin{tikzpicture}[scale=0.75]
    % 辺の長さの定義
    \def\a{3}
    \def\b{4}
    \def\c{5}
    
    % 大きな正方形の描画
    \draw (0,0) rectangle (\a+\b,\a+\b);
    
    % 4つの直角三角形の描画
    %第1の三角形
    \draw (0,0) -- (\b,0) -- (0,\a) -- cycle;
    \draw (0,\a) -- node[midway,above] {$c$} (\b,0);
    
    % 第2の三角形
    \draw (\b,0) -- (\a+\b,0) -- (\a+\b,\b) -- cycle;
    \draw (\a+\b,\b) -- node[midway,above] {$c$} (\b,0);
    
    % 第3の三角形
    \draw (\a+\b,\b) -- (\a+\b,\a+\b) -- (\a,\a+\b) -- cycle;
    \draw (\a,\a+\b) -- node[midway,above] {$c$} (\a+\b,\b);
    
    % 第4の三角形
    \draw (\a, \a+\b) -- (0, \a+\b) -- (0,\a) -- cycle;
    \draw (0,\a) -- node[midway,above] {$c$} (\a, \a+\b);
    
    % 中央の小さな正方形の描画
    \draw (\b,0) -- (\a+\b,\b) -- (\a,\a+\b) -- (0,\a) -- cycle;
    
    % 寸法のラベル
    \draw [<->] (-0.5,0) -- (-0.5,\a) node[midway,left] {$a$};
    \draw [<->] (-0.5,\a) -- (-0.5,\a+\b) node[midway,left] {$b$};
    \draw [<->] (0,-0.5) -- (\b,-0.5) node[midway,below] {$b$};
    \draw [<->] (\b,-0.5) -- (\a+\b,-0.5) node[midway,below] {$a$};
    
    \end{tikzpicture}
    \end{center}
    \end{example}

    \subsubsection{命題の反証と反例}

    数学において,「すべての$n$に対して$P(n)$が成り立つ」という全称命題が偽であることを証明するには,\textbf{反例}を一つ示すだけで十分である.これは,全称命題が「すべての場合において成り立つ」ことを主張しているため,一つでも例外があれば命題全体が偽となるからである.

    \begin{example}{反例による反証}{反例による反証}
        次の命題を考える:
        \[
        \forall n \in \mathbb{N}\, : \, \frac{n^3}{24} + 84 \geq n^2.
        \]
        この命題が成り立たないことを示すためには,ある$n$について不等式が成立しないことを示せばよい.実際に,$n = 15$のとき,
        \[
        \frac{15^3}{24} + 84 = \frac{3375}{24} + 84 \approx 140.625 + 84 = 224.625,
        \]
        一方,
        \[
        15^2 = 225.
        \]
        したがって,
        \[
        \frac{15^3}{24} + 84 \approx 224.625 < 225 = 15^2.
        \]
        このように,$n = 15$において不等式が成り立たないため,元の命題は偽であることが分かる.
    \end{example}

    また,全称命題が偽であることを証明する際には,「反例をどう見つけたか」ということは明記しない場合が多い.
    たとえば\exref{反例による反証}の場合に$n=15$を見つけるには,微分法を用いて議論したり様々な方法があるが,そのような「反例を見つけるまでの過程」を説明する必要はなく,「$n=15$の場合に命題が成り立たないこと」のみを証明として書けばよい.
    
    \phantomsection
    \subsubsection{様々な証明法}

    \begin{description}
        \item[帰納法] \mbox{} \\
        たとえば数学的帰納法は,数学的な主張が自然数全体に対して成り立つことを示すための証明法である.
        \item[背理法] \mbox{} \\
        「$P$が成り立たないと仮定すると矛盾が生じる」という論理的な構造を用いて証明を行う方法である.
        \item[直接証明] \mbox{} \\
        「$P$が成り立つことを示す」という形式で証明を行う方法である.
    \end{description}
    
    以下では,「$n \geq 2$ をみたす任意の自然数について,$2^n > n$である」という命題を3通りのやり方で証明してみよう.
    \begin{example}{帰納法}{帰納法}
      \begin{enumerate}[(I)]
      \item $n=2$のとき,
      \[
      2^2 = 4 > 2
      \]
      
      となり,与えられた不等式は成立する.
      \item 自然数$k$を$k \geqq 2$を満たすように任意にとる.$2^k > k$ が成り立つと仮定する.このとき,
      \[
      2^{k+1} = 2 \cdot 2^k > 2k.
      \]
      ここで,$k \geqq 2$であるから,$2k - (k + 1) = k - 1 \geq 1$であるがゆえに,$2k > k + 1$である.
      
      したがって,
      
      \[
      2^{k+1} > 2k > k + 1.
      \]
      ゆえに,$n=k+1$のときも$2^{k+1} > k + 1$である.
      \end{enumerate}
      
      以上の考察と数学的帰納法により,$2^n > n$ は $n \geq 2$をみたすすべての自然数について成り立つ.
      \end{example}

      \begin{example}{背理法}{背理法}
       $n \geqq 2$をみたす自然数で $2^n \leqq n$ となるものが存在すると仮定する.
        $2^n$ の正の約数の個数は,指数に $1$ を加えたものなので,$2^n$の正の約数の個数は $n + 1$ 個以上である.
        しかし,仮定より $2^n \leqq n$ なので,$2^n$ は $n$ 以下の数である.

        これは,$n$ 以下の数が $n + 1$ 個以上の正の約数を持つことを意味するが,これは矛盾である.
        
        したがって,仮定に矛盾が生じるため,$2^n > n$ が成り立つ.
      \end{example}

      \begin{example}{直接証明}{直接証明}
        \[
        2^n = n \int_{1}^{2} x^{n-1} \, dx + 1 
        \]
        となることはよい.さて,$x \in [1, 2]$ かつ $n \geqq 2$ なので,
        \[
        x^{n-1} \geqq x^{1} \geqq 1
        \]
        が成り立つ.したがって,
        \begin{align*}
        2^n &= n \int_{1}^{2} x^{n-1} \, dx + 1 \\
        & \geqq  n \int_{1}^{2} \, dx +1 \\
        & = n+1.
        \end{align*}
        これと$n+1 >n$であることを併せると,$n \geqq 2$のとき$2^n > n$である.
      \end{example}


\section{用語の定義}

\phantomsection
\subsection{定義・命題・定理・補題・系}

\phantomsection
\subsubsection{定義}

\emph{定義(definition)}とは,用語の意味を明確に述べたものであり,\textbf{Def}と略記される.同じ事柄について,2つ以上の定義の形式があることもある.次の例を見てみよう:

\begin{example}{「絶対値」の定義}{「絶対値」の定義}
	$ x \in \mathbb{R}$に対して,$x$の絶対値を$\abs{x}$とかき,次のように定義する:
	\[
	\abs{x} \coloneqq \max \{ x , -x \} .
	\]
	この定義は以下のような形式で表現してもよい:
	\[
	\abs{x} \coloneqq  \sqrt{x^2}.
	\]
\end{example}

次に,定義の性質についてみていこう.
線型代数の講義で学ぶことであるが,逆行列の定義は以下のようになる:

\begin{example}{「逆行列」の定義}{「逆行列」の定義}
	正方行列$A$に対して,$BA = AB =E$となるような$B$が存在するとき,このような$B$を$A$の逆行列という.
\end{example}


ここで注意するのは \emph{数学では定義は最小限の情報にとどめることが慣習となっている}ということである.たとえば,\exref{「逆行列」の定義}の定義から,以下の命題\footnote{「命題」については,のちほど詳しくみていくとする.ここでは「証明を与えるべきである主張」という理解でよい.}が成り立ち,その証明は容易である.

\begin{prop}{逆行列の一意性}{逆行列の一意性}
	正方行列$A$に対して,$A$の逆行列は存在するとすればただひとつである.
\end{prop}

\begin{tleftbar}
	\begin{proof}
	\prref{逆行列の一意性}の証明は,$A$の逆行列が$B$,$C$であるとすると,
	\[
		AB=BA=E,\quad AC=CA=E
	\]
	が成り立つ.このとき,
	\[
		B=BE=B(AC)=(BA)C=EC=C
	\]
	が成り立つ.よって,$B=C$であり,逆行列の一意性が示された.
	\end{proof}
\end{tleftbar}

このことから,\exref{「逆行列」の定義}でとりあげた逆行列の定義をもっと詳しく

\begin{quote}
	正方行列$A$に対して,$BA = AB =E$となるような$B$が存在するとき,このような$B$はただひとつで,$B$を$A$の逆行列という.
	$B$は一意に存在するので,これを$A^{-1}$と記す\footnote{一意に存在することがわかっていないと,$A^{-1}$のような記法で表すことはためらわれる.}.もちろん
	\[
		AA^{-1}=A^{-1}A=E.
	\]
\end{quote}
と,「逆行列の一意性を定義に含めてもいいのではないか.」と主張する人がいるかもしれない.
ただ,数学では「定義は最小限の情報にとどめ,そこから導かれる主張を命題として証明する」という慣習があり,
なにが「定義」で,なにが「証明すべきこと」であるかはっきりさせることが多い\footnote{ただ,実際はここで取り上げた逆行列の定義も情報過多である.線型代数で学ぶことになるが,$AB=E$と$BA=E$のどちらか片方の式のみで定義していいからである.}.

\phantomsection
\subsubsection{命題}

\emph{命題(proposition)}とは,真偽が定まっている文を指す\footnote{命題はふたつの意味があり,ここでいう命題は「真偽が決まっている文」というニュアンスを持つ「広い意味での命題」というよりかは「定理・命題・補題」などと並列して表記される「狭い意味での命題」である.「狭い意味での命題」は正しい主張である.
	たとえば,「$3$以上の自然数$n$に対して$x^n + y^n =z^n$は自然数解を持たない」という「フェルマーの最終定理」は,
	アンドリュー・ワイルズが証明するまでは「真偽が決まっているがどちらかはわからない」という「広い意味での命題」であったが,
	証明されたのちに「広い意味での命題」であると同時に「狭い意味での命題」にもなった.}.\textbf{Prop}と略記される.
このような論理体系を\emph{二値論理}という.
二値論理においては,命題は「真」か「偽」のどちらかあり,命題が真であることを$\mathrm{T}$,偽であることを$\mathrm{F}$と表す.

中間として扱われる主張には
\begin{enumerate}[(1)]
	\item 定義が曖昧なもの \label{enu:定義が曖昧なもの}
	\item 意味が曖昧なもの \label{enu:意味が曖昧なもの}
	\item パラドックス \label{enu:パラドックス}
\end{enumerate}
などがある.\ref{enu:定義が曖昧なもの},\ref{enu:意味が曖昧なもの}についてはのちほど説明するとして,ここでは\ref{enu:パラドックス}について例をあげよう.

\begin{example}{自己言及のパラドックス}{自己言及のパラドックス}
	次のような主張を考える:
	\begin{quote}
		この文は偽である.
	\end{quote}
	この主張は,自己言及のパラドックスであり,真偽が決まらない.
	
	以下も自己言及のパラドックスの例である:
	\begin{quote}
		「この壁に貼り紙をしてはならない」と書かれた貼り紙
	\end{quote}
\end{example}

ここまでで,二値論理でとりあげない主張を述べてきたが,そろそろ二値論理で取り扱う主張のお話に戻ろう.
また,以下では簡単のために,「広い意味での命題」と「狭い意味での命題」のどちらも「命題」と記すと約束する.


\begin{example}{命題}{命題}
    次に示す文は真偽が真の命題である.
    \begin{enumerate}[(1)]
        \item 「1+1=2」
        \item 「 23 > 17」
        \item 「円周率は100未満である」
        \item 「信号の色は3色である」
        \item 「霞ヶ浦は日本で二番目に面積が大きい湖である」
    \end{enumerate}
    また,次に示す文は真偽が偽の命題である.
    \begin{enumerate}[(a)]
        \item 「1+1=46」
        \item 「23>26」
        \item 「円周率は3未満である」
        \item 「信号の色は5色である」
        \item 「霞ヶ浦は日本で五番目に面積が大きい湖である」
    \end{enumerate}
\end{example}

\begin{example}{命題でない文}{命題でない文}
    次に示す文は命題ではない
    \begin{enumerate}[(A)]
    \item 「$1+1$」(なにも主張しておらず,真か偽か判定できない)
    \item 「霞ヶ浦の面積は大きい」(客観的に大きいか判定できない)
    \item 「桃はおいしい」(基準が明確でなく,客観的に真か偽か判定できない)
    \item 「$x^2 > 4$」($x$に具体的な値を代入しないと真か偽か判定できない)
    \end{enumerate}
\end{example}

\phantomsection
\subsubsection{定理}

\emph{定理(theorem)}とは,正しいと分かっている数学の主張\footnote{つまり,「狭い意味での命題」のこと.}の中でもとりわけ重要なものを指す.\textbf{Thm}と略記される.


定理は数学における真理であり,新たな理論の構築や他の結果の証明の基礎となる.定理は厳密な論理的推論と証明によって裏付けられており,その証明は既知の定理,定義,公理,または論理的推論規則に基づいて行われる.

定理が重要な主張であるがゆえに,「余弦定理」,「加法定理」,「ハイネ・ボレルの被覆定理」など,固有の名前が与えられているものも存在する.
その固有の名前は,定理の主張の詳細,もしくは発見者の名前ににちなんで付けられることが多い.例えば,「三平方の定理」や「コーシー・シュワルツの不等式」などである.

\begin{example}{三平方の定理}{三平方の定理}
	三平方の定理の主張は以下のようになる:
	\begin{quotation}
    直角三角形の斜辺の長さを$c$とし,他の二辺の長さを$a$,$b$としたとき,
    \[
        a^2+b^2=c^2
    \]
    が成り立つ.
\end{quotation}
\end{example}
  
  \begin{example}{コーシー・シュワルツの不等式}{コーシー・シュワルツの不等式}
  コーシー・シュワルツの不等式は,内積空間における基本的な不等式であり,以下のように表される:
  \begin{quotation}
  任意の内積空間において,ベクトル$u$と$v$に対して,
  \[
  \norm{\langle u, v \rangle} \leq \norm{u} \cdot \norm{v}
  \]
  が成り立つ.
  \end{quotation}
  この不等式は解析学や線型代数学など,多くの分野で重要な役割を果たす.
  \end{example}
  
  
  \begin{example}{微分積分学の基本定理}{微分積分学の基本定理}
  この定理は微分と積分の関係を明らかにするものであり,以下のように述べられる:
  \begin{quotation}
  $f$を区間$[a, b]$上で連続な実数値関数とする.このとき,
  \[
  F(x) = \int_a^x f(t) \, dt
  \]
  と定義すると,$F$は$[a, b]$上で微分可能であり,
  \[
  F'(x) = f(x)
  \]
  が成り立つ.
  \end{quotation}
  この定理により,積分と微分が互いに逆操作であることが示される.
  \end{example}
  \phantomsection
\subsubsection{補題・系}

\emph{補題(lemma)}とは,定理や命題を証明する際に,その証明の一部として利用される補助的な命題のことを指す.\textbf{Lem}と略記される.補題は,直接的に重要な結果でない場合もあるが,より複雑な定理を証明するための重要なステップとなる主張である.

一方,\emph{系(corollary)}とは,既に証明された定理や命題から直接的に導かれる結果のことである.\textbf{Cor}と略記される.系は先の結果を応用することで容易に得られる新たな命題であり,元の定理の応用例や特別な場合を示すことが多い.

補題や系を説明するためには,関連するいくつかの命題や定理が必要である.以下に具体的な例を示す.

\begin{example}{ユークリッドの補題}{ユークリッドの補題}
ユークリッドの補題は,整数論における基本的な結果であり,素因数分解の一意性を証明する際に重要な役割を果たす.その主張は以下の通りである:

\begin{quote}
素数$p$が整数$a$と$b$の積$ab$を割り切るならば,$p$は$a$と$b$の少なくとも一方を割り切る.
\end{quote}

\end{example}

\begin{tleftbar}
\begin{proof}
素数$p$が$ab$を割り切るとする.もし$p$が$a$を割り切らないならば,$\gcd(p, a) = 1$である.
このとき,ベズーの等式より,ある整数$s$と$t$が存在して,$sp + ta = 1$が成り立つ.
両辺に$b$を掛けると,$spb + tab = b$となる.
左辺の$spb$は$p$で割り切れるが,$tab$は$p$で割り切れるため,
右辺の$b$も$p$で割り切れる.したがって,$p$は$b$を割り切る.
\end{proof}
\end{tleftbar}

この補題を用いて,素因数分解の一意性(算術の基本定理)を証明することができる.

\begin{example}{自然数の約数の個数}{自然数の約数の個数}
自然数$n$の正の約数の個数$d(n)$は,$n$の素因数分解に基づいて以下の式で与えられる:

\begin{equation}
d(n) = (e_1 + 1)(e_2 + 1) \dotsm (e_k + 1)
\end{equation}

ただし,$n$を素因数分解して$n = p_1^{e_1} p_2^{e_2} \dotsm p_k^{e_k}$と表した.
\end{example}

\begin{tleftbar}
\begin{proof}
各素因数$p_i$について,指数$e_i$は$0$から$e_i$までの値を取り得る.
したがって,各$p_i$に対する約数の取り得る指数の個数は$e_i + 1$個である.
全ての素因数について独立に指数を選ぶことで,
$n$の全ての約数を生成できるため,約数の総数は各$(e_i + 1)$の積となる.
\end{proof}
\end{tleftbar}

この命題は,素因数分解の一意性から直接的に導かれる結果である.

\begin{example}{約数の個数が奇数となる条件}{約数の個数が奇数となる条件}
自然数$n$の約数の個数$d(n)$が奇数となるための必要十分条件は,$n$が完全平方数であることである.
\end{example}

\begin{tleftbar}
\begin{proof}
約数の個数$d(n) = (e_1 + 1)(e_2 + 1) \dotsm (e_k + 1)$である.
各$(e_i + 1)$が奇数となるためには,$e_i$が偶数でなければならない.
つまり,全ての素因数の指数$e_i$が偶数であるとき,$n$は各素因数の偶数乗の積であり,
これは$n$が完全平方数であることと同値である.
逆に,$n$が完全平方数であれば,各$e_i$は偶数であり,したがって$d(n)$は奇数となる.
\end{proof}
\end{tleftbar}

\phantomsection
\subsection{数学における言い回し}
\phantomsection
\subsubsection{存在}



数学の証明において,「存在」は重要な概念である.といっても,あまりこのことを意識したことのない読者の方もいると思うので,この場を借りて具体例をもとに説明を試みることとする.

まず,「最大値・最小値の定理」を考えてみる.この定理は,
\begin{quotation}
	$[a,b]$で連続な関数$f$に対して,$f$は$[a,b]$上で最大値と最小値を持つ.
\end{quotation}
というものある.

「最大値・最小値が存在するなんて当たり前だ」と思われる読者もいるかもしれない.しかし,本当にそれは自明なのか.
実際には,関数の連続性や区間の閉有界性といった条件が揃って初めて,最大値や最小値の「存在」を保証できる.この定理を証明するにあたっては,厳密な数学的議論が必要となる.

次に,指数関数を考える際に有理数列$(x_n)_{n \in \mathbb{N}}$を用いる場合の例を見てみよう:

$x \in \mathbb{R}$としたとき,$a^x$を定義するために,
\[
	\lim_{n \to \infty} a^{x_n} = a^x
\]

とするが,この極限が存在することは本当に自明なのか.
$a^2$や$a^{2/3}$の具体的な値のイメージは思い浮かぶと思うが,たとえば$a^{\pi}$のイメージについてはどうであろうか.

このように,極限の存在を証明するためには,有理数列の収束性など,細かな数学的性質を確認する必要がある.このようにして初めて,指数関数の定義が厳密なものとなるのである.

\subsubsection{一意性}

数学において「一意性」が重要である場面は多い.読者の中には線型代数の講義で「逆行列の一意性」などに触れた方もいると思われる.なぜ重要であるのか,一つ例を挙げて考えてることとする.

微分積分の講義で習う定理に「平均値の定理」というものがある.
その主張は
\begin{quotation}
	$[a,b]$で連続,$(a,b)$で微分可能な関数$f$に対して,
	\[
	\frac{f(b)-f(a)}{b-a}=f'(c)
	\]
	をみたす$ c \in (a,b)$が存在する.
\end{quotation}

というものである.証明はのちに述べるとして,この定理の主張を少し変更してみよう:

\begin{quotation}
	$[a,b]$で連続,$(a,b)$で微分可能な関数$f$に対して,
	\[
	\frac{f(b)-f(a)}{b-a}=f'(c)
	\]
	をみたす$ c \in (a,b)$がただひとつ存在する.
\end{quotation}

ここでは「$c \in (a,b)$が存在する」という主張を「$c \in (a,b)$がただひとつ存在する」というより強い主張に変更した.この主張の真偽は偽である.以下で,このことが問題になる状況を挙げよう.

\begin{wrapfigure}[11]{r}{0.55\textwidth}
	\centering
\begin{tikzpicture}
    \begin{axis}[
        axis lines=middle,
        xmin=-1.2, xmax=1.2,
        ymin=-1.5, ymax=1.5,
        xtick=\empty, ytick=\empty, % 目盛を消す
        xlabel={}, ylabel={},       % ラベルを消す
        samples=200,
        domain=-1.2:1.2,
        width=10cm,
        height=8cm,
    ]
    % 関数 f(x) = x^3 のグラフ
    \addplot [dblue, thick] {x^3};

    % 点 (-1, -1) と (1, 1) を結ぶ割線
    \addplot [dred, thick] coordinates {(-1, -1) (1, 1)}; 

    % 平均値の定理を満たす点 x = -1/√3 の接線
    \def\xone{-1/sqrt(3)}
    \def\yone{(\xone)^3}
    \addplot [dgreen, dashed] {1*(x - \xone) + \yone};

    % 平均値の定理を満たす点 x = 1/√3 の接線
    \def\xtwo{1/sqrt(3)}
    \def\ytwo{(\xtwo)^3}
    \addplot [dgreen, dashed] {1*(x - \xtwo) + \ytwo};

    % 重要な点をマーク
    \addplot [only marks, mark=*, mark options={fill=black}] coordinates {(\xone, \yone) (\xtwo, \ytwo)};
    \addplot [only marks, mark=*, mark options={fill=dred}] coordinates {(-1, -1) (1, 1)}; % 赤い丸に変更
    \end{axis}
\end{tikzpicture}
\end{wrapfigure}
$f(x)=x^3$という関数を考える.この関数は$[-1,1]$で連続,$(-1,1)$で微分可能である.

このとき,図のように,条件を満たす$c \in (-1,1)$は複数存在するので,「ただひとつ存在する」という主張は偽である.
さらに言えば,2本より多くこのような接線を引ける場合もある.

このことから,安易に「ただ一つ存在する」などと強い主張をすることは避けるべきであることがわかる.このことは平均値の定理に限らず,中間値の定理なども同様である.


\phantomsection
\subsubsection{かつ/または}



数学と日常における「または」の使い方は異なる.以下に例を挙げよう.

\begin{example}{「または」の使用例}{「または」の使用例}
    \begin{enumerate}[(A)]
	\item ランチメニューの主食として,米またはパンがついてくる\footnote{この文を「米とパンの両方が食べられる」と解釈してもらっては困る.}.\label{enu:米またはパン}
	\item 「運転免許を持っていない人」または「18歳未満の人」はレンタカーを借りることができない. \label{enu:運転免許または18歳未満}
	\end{enumerate}
\ref{enu:米またはパン}は「どちらか片方のみ」の意味で「または」を使い,\ref{enu:運転免許または18歳未満}は「いずれかが」の意味で「または」を用いている.
\end{example}

数学では,「または」は「いずれかが」の意味で使われ,「どちらか片方のみ」の意味で使われることはない.

たとえば,$A$,$B$を集合とするとき,
\[
x \in A \cup B
\]
は「$x$は$A$の元であるか,$B$の元であるか,あるいは両方である」という意味である.
つまり,「$ x \in A$であり,$x \notin  B$である」あるいは「$ x \notin A$であり,$x \in B$である」といった状況のときにも,$x \in A \cup B$と記す.
\newpage 

\cite{nakajima},\cite{kaneko}を参考にした.

\printbibliography[title=参考文献]
\end{document}