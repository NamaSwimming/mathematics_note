
\usepackage{titlesec}
\usepackage{picture}
% ---- 行の高さに自動追従する左肩の細四角 ----
\newcommand{\TitleSquareToText}[2][1.05ex]{%
  \begingroup
    \leavevmode
    \setbox0=\hbox{\strut}%
    \textcolor{#2}{\rule[-\dp\strutbox]{#1}{\dimexpr\ht\strutbox+\dp\strutbox\relax}}%
  \endgroup
}

% ---------------- Part(下線=フル幅) ----------------
\titleformat{\part}[display]
  {\normalfont\LARGE\bfseries\centering\color{appleInk}}
  {\small\scshape\color{appleGray}\partname~\thepart}
  {0.3ex}
  {}%
  [{%
     \vspace{0.6ex}%
     \noindent{\color{appleLine}\rule{\linewidth}{1.2pt}}%
     \nobreak\vspace{0.25ex}%
     \noindent{\color{appleLine}\rule{\linewidth}{0.6pt}}%
  }]
% ---------------- Section(下線=66%) ----------------
\titleformat{\section}[block]
  {\normalfont\Large\bfseries\color{appleInk}}
  {}{0pt}
  {%
    \TitleSquareToText[1.05ex]{SectionTag}\hspace{0.6em}%
    \thesection\quad
  }%
  [{%
     \vspace*{-0.48\baselineskip}% ← 線を上に引き上げて詰める
     \noindent{\color{appleLine}\rule{0.66\linewidth}{0.8pt}}%
     \nobreak\vspace{-0.10\baselineskip}% ← 線の下側(本文との距離)
  }]

% ---------------- Subsection(下線=50%) ----------------
\titleformat{\subsection}[block]
  {\normalfont\large\bfseries\color{appleInk}}
  {}{0pt}
  {%
    \TitleSquareToText[0.95ex]{SubsectionTag}\hspace{0.55em}%
    \thesubsection\quad
  }%
  [{%
     \vspace*{-0.52\baselineskip}% ← もう少しだけ控えめに詰める
     \noindent{\color{appleLine}\rule{0.50\linewidth}{0.6pt}}%
     \nobreak\vspace{-0.18\baselineskip}%
  }]

% ---------------- Subsubsection(下線=42%) ----------------
\titleformat{\subsubsection}[block]
  {\normalfont\large\bfseries\color{appleInk}}
  {}{0pt}
  {%
    \TitleSquareToText[0.90ex]{SubsubsectionTag}\hspace{0.50em}%
    \thesubsubsection\quad
  }%
  [{%
     \vspace*{-0.40\baselineskip}% ← さらに控えめ
     \noindent{\color{appleLine}\rule{0.42\linewidth}{0.5pt}}%
     \nobreak\vspace{-0.16\baselineskip}%
  }]

