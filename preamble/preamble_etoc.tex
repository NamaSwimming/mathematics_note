
% ===================== ToC (etoc) — Apple風見出し =====================
\makeatletter
\AtBeginDocument{%

  % 目次に出す深さ
  \etocsetnexttocdepth{subsubsection}

  % 目次ブロックの前後
  \etocsettocstyle{%
    \par\vspace{10pt}%
    \noindent\hspace*{-\leftskip}%
    \begin{tikzpicture}[x=\linewidth, baseline]
      \draw[line cap=round, line width=0.8pt, color=appleLine] (0,0) -- (1,0);
      \node[inner sep=2.5pt, outer sep=0pt, fill=appleCard] at (0.5,0)
        {\bfseries\Large\textcolor{appleInk}{目\kern0.35em 次}};
        % 背景が白なら fill=white に変更
    \end{tikzpicture}%
    \vspace{8pt}%
  }{%
    \vspace{8pt}%
    \noindent\hspace*{-\leftskip}%
    \begin{tikzpicture}[x=\linewidth, baseline]
      \draw[line cap=round, line width=0.5pt, color=appleLine] (0,0) -- (1,0);
    \end{tikzpicture}%
    \vspace{12pt}%
  }

  % ---- 各レベルの表示 ----
  \etocsetstyle{part}
    {\noindent}
    {%
      \vskip20pt
      \parindent0pt \leftskip0pt
      \etoclink{\bfseries\huge\color{appleInk}\etocname}\par
      \vskip10pt
    }
    {}

  \etocsetstyle{section}
    {\noindent}
    {%
      \vskip10pt
      \parindent0pt \leftskip5pt
      \etoclink{\bfseries\LARGE\color{appleInk}\etocnumber~\etocname}%
      \leaders\hbox to .5em{\hss.\hss}\hfill \etocpage\par
      \vskip10pt
    }
    {}

  \etocsetstyle{subsection}
    {\noindent}
    {%
      \vskip6pt
      \parindent0pt \leftskip25pt % ←調整可(例: 20pt)
      \etoclink{\bfseries\large\textcolor{appleInk}{\etocnumber~\etocname}}%
      \nobreak\leaders\hbox to .5em{\hss.\hss}\hfill \etocpage\par
      \vskip6pt
    }
    {}



\etocsetstyle{subsubsection}
{\noindent}
{%
  \parindent0pt \leftskip50pt % `subsubsection` は30ptのインデント
  \etoclink{\bfseries\normalsize\color{appleInk}{\etocnumber~\etocname}}%
  \ % 項目の間に半角スペースを挿入
  (p.\etocpage)% ページ番号の表示
  \hspace{1em}
}
{}
}
\makeatother
% ======================================================================
