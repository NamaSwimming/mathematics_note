% tcolorbox系
\usepackage[many]{tcolorbox}
\tcbuselibrary{breakable,skins,theorems,xparse,hooks}
\newtcolorbox{hosoibox}[1]{colframe=black,colback=white,coltitle=black,colbacktitle=white,boxrule=0.5pt,arc=0mm,enhanced,attach boxed title to top left={xshift=10mm,yshift=-3mm},boxed title style={frame hidden},title=#1}
\usepackage{lipsum}

%leftbar環境に注釈が入れられないことを解消する環境.名前は,tcolorboxの[t]とleftbarの組み合わせ
\newtcolorbox{tbleftline}{blanker,breakable,left=5mm,borderline west={1.1mm}{0pt}{black}}
\newenvironment{tleftbar}{\begin{tbleftline}\setlength{\parindent}{1\zw}}{\end{tbleftline}}

%leftbar環境に注釈が入れられないことを解消する環境.名前は,tcolorboxの[t]とleftbarの組み合わせ
\newtcolorbox{ttbleftline}{blanker,breakable,left=5mm,borderline west={1.1mm}{0pt}{darkgray}}
\newenvironment{ttleftbar}{\begin{ttbleftline}\setlength{\parindent}{1\zw}}{\end{ttbleftline}}


%dotleftbar環境
\newtcolorbox{tbdotleftline}{%
  blanker, left=5mm,breakable,
  borderline west={1.1mm}{0pt}{draw=black, dotted}
}
\newenvironment{dotleftbar}{%
  \begin{tbdotleftline}\setlength{\parindent}{1\zw}%
}{%
  \end{tbdotleftline}%
}
% 必要パッケージ
% \usepackage[most]{tcolorbox}   % 既に読み込み済みなら不要
% \tcbuselibrary{skins,breakable}

% ------------------------------------------------
% Apple調の落ち着いた共通ベース
% ------------------------------------------------
% color_style.tex にある想定色(例):
% appleCard, applePaper, appleInk, appleInkGray, appleGrayLight, appleGray, appleGrayDark, applePurple
% ※ applePurple が無ければ,appleInkGray へ一時的に差し替えて OK

% アクセント用の色エイリアス(ここだけ触れば配色一括変更できます)
\colorlet{acc@th}{appleInkGray}                 % Theorem:最もニュートラル
\colorlet{acc@pr}{appleGrayDark}                % Proposition:やや濃い灰
\colorlet{acc@co}{appleGray}                    % Corollary:中間灰
\colorlet{acc@ax}{appleInk}                     % Axiom:ほぼ黒(最も重い主張)
\colorlet{acc@de}{applePurple}                  % Definition:紫(主張はするが落ち着き)
\colorlet{acc@le}{applePurple!75!appleInk}      % Lemma:定義より少し落ち着いた紫
\colorlet{acc@ex}{appleGrayLight}               % Example:軽めの灰(導入・補助)
\colorlet{acc@qu}{applePurple!60!appleGray}     % Question:紫寄り灰

% 共通ベース:白カード+左2ptアクセント+下端0.8ptの薄線
\tcbset{
  theo-base/.style={
    enhanced, breakable, sharp corners,
    boxrule=0pt, no shadow,
    colback=appleCard, colframe=appleCard, colbacktitle=appleCard,
    coltitle=appleInk, coltext=appleInk,
    fonttitle=\gtfamily\sffamily\bfseries\upshape,
    left*=1\zw, right*=1\zw, top=8pt, bottom=10pt,
    borderline south={0.8pt}{-2pt}{appleLine}, % 「箱のすぐ下」に薄い仕切り線
    before skip=16pt, after skip=18pt,
    before upper={\setlength{\parindent}{1\zw}},
    before lower={\setlength{\parindent}{1\zw}}
  },
  % 1 引数: アクセント色
  mytheo/.style n args={1}{
    theo-base,
    borderline west={2pt}{0pt}{#1} % 左に2ptのアクセントバー
  },
}

% ------------------------------------------------
% 環境定義(番号は subsection 内で進む)
% ------------------------------------------------
\newtcbtheorem[number within=subsection]{theorem}{Theorem}%
{mytheo={acc@th}}{th}
\newcommand{\thref}[1]{\textsf{\bfseries Theorem~\ref{th:#1}}}

\newtcbtheorem[use counter from=theorem]{prop}{Proposition}%
{mytheo={acc@pr}}{pr}
\newcommand{\prref}[1]{\textsf{\bfseries Proposition~\ref{pr:#1}}}

\newtcbtheorem[use counter from=theorem]{cor}{Corollary}%
{mytheo={acc@co}}{co}
\newcommand{\coref}[1]{\textsf{\bfseries Corollary~\ref{co:#1}}}

\newtcbtheorem[use counter from=theorem]{axiom}{Axiom}%
{mytheo={acc@ax}}{ax}
\newcommand{\axref}[1]{\textsf{\bfseries Axiom~\ref{ax:#1}}}

\newtcbtheorem[use counter from=theorem]{definition}{Definition}%
{mytheo={acc@de}}{de}
\newcommand{\deref}[1]{\textsf{\bfseries Definition~\ref{de:#1}}}

\newtcbtheorem[use counter from=theorem]{lemma}{Lemma}%
{mytheo={acc@le}}{le}
\newcommand{\lemref}[1]{\textsf{\bfseries Lemma~\ref{le:#1}}}
% 互換:もし既存で \leqqref を使っていたら壊さないように
\providecommand{\leqqref}[1]{\lemref{#1}}

\newtcbtheorem[use counter from=theorem]{example}{Example}%
{mytheo={acc@ex}}{ex}
\newcommand{\exref}[1]{\textsf{\bfseries Example~\ref{ex:#1}}}

\newtcbtheorem[use counter from=theorem]{question}{Question}%
{mytheo={acc@qu}}{qu}
\newcommand{\quref}[1]{\textsf{\bfseries Question~\ref{qu:#1}}}

% その他の設定
% 囲い枠
\DeclareTColorBox{simplesquarebox}{ o m O{.5} O{} }% 
    {empty, left=2mm, right=2mm, top=-1mm, attach boxed title to top left={xshift=1.2\zw},
    boxed title style={empty,left=-2mm,right=-2mm}, colframe=black, coltitle=black, coltext=black, breakable,  
    underlay unbroken={\draw[black,line width=#3pt]
        (title.east) -- (title.east-|frame.east) -- (frame.south east) -- (frame.south west) -- (title.west-|frame.west) -- (title.west); },
    underlay first={\draw[black,line width=#3pt](title.east) -- (title.east-|frame.east) -- (frame.south east) ;
        \draw[black,line width=#3pt] (frame.south west) -- (title.west-|frame.west) -- (title.west); },
    underlay middle={\draw[black,line width=#3pt](frame.north east) -- (frame.south east) ;
        \draw[black,line width=#3pt](frame.south west) -- (frame.north west) ;},
    underlay last={\draw[black,line width=#3pt](frame.north east) -- (frame.south east) -- (frame.south west) -- (frame.north west) ;},
    fonttitle=\gtfamily, IfValueTF={#1}{title=【#2】〈#1〉}{title=【#2】},#4}


    \colorlet{colexam}{red!75!black}
  \newtcolorbox[auto counter,number format=\Roman]{mycolumn}{
      empty,
      title={\bfseries\sffamily Column \thetcbcounter}, % カウンタをローマ数字で表示
      attach boxed title to top left,
      boxed title style={
        empty,
        size=minimal,
        toprule=2pt,
        top=4pt,
        bottom=3pt,
        left=1cm, % タイトルを右に移動
        overlay={
          % タイトルの上の線を削除
          % \draw[colexam,line width=2pt]
          %   ([yshift=-1pt]frame.north west) -- ([yshift=-1pt]frame.north east);
        }
      },
      coltitle=colexam,
      fonttitle=\Large\bfseries,
      before=\par\medskip\noindent,
      parbox=false,
      boxsep=0pt,
      left=5mm, % 左のマージンを増やしてタイトルを右に移動
      right=3mm,
      top=4pt,
      breakable,
      pad at break*=0mm,
      vfill before first,
      overlay unbroken={
        \draw[colexam,line width=2pt]
          ([xshift=-0.5pt,yshift=16pt]frame.north east)
          -- ([xshift=-0.5pt]frame.south east);
        \draw[colexam,line width=2pt]
          ([xshift=-1pt,yshift=16pt]frame.north west)
          -- ([xshift=-1pt]frame.south west);
      },
      overlay first={
        \draw[colexam,line width=2pt]
          ([xshift=-0.5pt,yshift=16pt]frame.north east)
          -- ([xshift=-0.5pt]frame.south east);
        \draw[colexam,line width=2pt]
          ([xshift=-1pt,yshift=16pt]frame.north west)
          -- ([xshift=-1pt]frame.south west);
      },
      overlay middle={
        \draw[colexam,line width=2pt]
          ([xshift=-0.5pt,yshift=16pt]frame.north east)
          -- ([xshift=-0.5pt]frame.south east);
        \draw[colexam,line width=2pt]
          ([xshift=-1pt,yshift=16pt]frame.north west)
          -- ([xshift=-1pt]frame.south west);
      },
      overlay last={
        \draw[colexam,line width=2pt]
          ([xshift=-0.5pt,yshift=16pt]frame.north east)
          -- ([xshift=-0.5pt]frame.south east);
        \draw[colexam,line width=2pt]
          ([xshift=-1pt,yshift=16pt]frame.north west)
          -- ([xshift=-1pt]frame.south west);
      },%
    }




% 例題環境
\newcounter{mondaibangou}
\newtcolorbox{mondai}[1][]{enhanced,boxrule=0.5mm,
        top=2pt,left=44pt,right=4pt,bottom=2pt,arc=0mm,
        colframe=blue!30!gray,
        boxrule=1pt,
        underlay={
        \node[inner sep=1pt,blue!50!black,fill=blue!10!white]at ([xshift=22pt,yshift=-9pt]interior.north west) {\stepcounter{mondaibangou}\bfseries\gtfamily 問題\themondaibangou};},
        segmentation code={%
        \draw[dashed] (segmentation.west)--(segmentation.east);
        \node[inner sep=1pt,blue!50!black,fill=blue!10!white] at ([xshift=22pt,yshift=-8pt]segmentation.south west) {\bfseries\gtfamily 解答};},
        before upper={\setlength{\parindent}{1\zw}},
        before lower={\setlength{\parindent}{1\zw}},#1
}

% 感想環境
\DeclareTColorBox{kans}{ o m O{3} O{}}%
{enhanced, colback=white, colframe=white,
attach boxed title to top left={xshift=1cm,yshift=-\tcboxedtitleheight/2}, fonttitle=\bfseries,varwidth boxed title=0.85\linewidth, coltitle=black, fonttitle=\gtfamily, 
enlarge top by=2mm, enlarge bottom by=2mm, breakable, sharp corners,
boxed title style={colback=white,left=0mm,right=0mm}, 
borderline={.75pt}{#3pt}{black,dotted},
% underlay settings...
IfValueTF={#1}{title=【#2】〈#1〉}{title=【#2】},#4}
