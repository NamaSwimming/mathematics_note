\usepackage{xcolor}
% colors
\definecolor{myblue}{HTML}{3300CC}
\definecolor{myred}{HTML}{CC33CC}
\definecolor{mypink}{HTML}{FF66FF}
\definecolor{mygreen}{HTML}{005500}
\definecolor{mynamecolor}{HTML}{5507FF}
\definecolor{myemailcolor}{HTML}{FF4F02}
\definecolor{darkbrown}{HTML}{880000}
\definecolor{dred}{HTML}{DD0000}
\definecolor{dgreen}{HTML}{005500}
\definecolor{dblue}{HTML}{000088}
\definecolor{mypurple}{HTML}{ba55d3} % 追加
\definecolor{burgundy}{rgb}{0.5, 0.0, 0.13}
\definecolor{charcoal}{rgb}{0.21, 0.27, 0.31}
\colorlet{lightmypurple}{mypurple!20!white}


% --- Base (Neutrals) ---
\definecolor{applePaper}{HTML}{F5F5F7} % ペーパー/セクション背景(明るい灰)
\definecolor{appleCard} {HTML}{FFFFFF} % 白カード
\definecolor{appleInk}  {HTML}{1D1D1F} % 本文インク(ほぼ黒)
\definecolor{appleInkGray}{HTML}{37373A} % Ink寄りのダークグレイ(要求により必須)
\definecolor{appleLine} {HTML}{D2D2D7} % 区切り線(Appleの薄い線)
\definecolor{appleGrayLight}{HTML}{E5E5EA} % 薄い灰(背景の差別化)
\definecolor{appleGray}     {HTML}{8E8E93} % 中間灰(注釈や副次情報)
\definecolor{appleGrayDark} {HTML}{3A3A3C} % 濃い灰(強めのUI線)
\definecolor{appleBlack}    {HTML}{1D1D1F} % 黒系のエイリアス

% --- Accents (Blue / Indigo / Purple / Magenta) ---
\definecolor{appleBlue600}{HTML}{0071E3} % Appleのボタン系ブルー
\definecolor{appleBlue500}{HTML}{0A84FF} % iOS System Blue
\definecolor{appleBlue300}{HTML}{64D2FF} % 薄いブルー(図形の塗りやハイライト)
\definecolor{appleIndigo}  {HTML}{5E5CE6} % iOS System Indigo
\definecolor{applePurple}  {HTML}{BF5AF2} % iOS System Purple(鮮やかな紫)
\definecolor{applePurpleDark}{HTML}{8E43E7} % 濃い紫(強調用)
\definecolor{appleMagenta} {HTML}{D6006D} % 赤紫(強調・リンク代替)
\definecolor{applePink}    {HTML}{FF2D55} % ピンク寄り赤紫(軽いアクセント)
\definecolor{appleOrange}  {HTML}{C76A00} % Dark Amber(白地でも視認性◎)
\definecolor{appleRed}     {HTML}{C92A2A} % Dark Red(物理の強調・白地でも視認性◎)


% --- 既存色名 → Apple調へのエイリアス(既存TikZコード互換) ---
\colorlet{Ink}{appleInk}                   % 既存 'Ink' を本文インクに
\colorlet{darkgray2}{appleGrayLight}       % 既存 'darkgray2' を薄い灰に
\colorlet{appleBlue}{appleBlue600}         % 後方互換のため残す(定義名そのまま)
\colorlet{appleLineDark}{appleGrayDark}    % 濃い線を統一

% --- ドキュメントで参照される補助色(必要なら継続使用) ---
\definecolor{appleTitleGray}{RGB}{86,86,91}
\definecolor{appleLineGray}{RGB}{134,134,139}

% --- 上下和(ハッチ/塗り) ---
\definecolor{CupertinoBlue}{HTML}{0A84FF}   % System Blue
\definecolor{CupertinoIndigo}{HTML}{5E5CE6} % System Indigo
\colorlet{upperfill}{CupertinoBlue!28}
\colorlet{lowerfill}{CupertinoIndigo!28}
% ---- モノトーンのタグ色 ----
\colorlet{SectionTag}{appleGrayDark}
\colorlet{SubsectionTag}{appleGray}
\colorlet{SubsubsectionTag}{appleInkGray}