\phantomsection
\part*{数学の基本}
\section{用語の定義}

\phantomsection
\subsection{定義・命題・定理・補題・系}

\phantomsection
\subsubsection{定義}

\emph{定義}\index{ていぎ@定義}(\emph{definition})とは,用語の意味を明確に述べたものであり,\textbf{Def}と略記される.
数学のお話にうつる前に,日常的な「定義」の例を見てみよう.

\begin{example}{アレニウスによる酸・塩基の定義}{アレニウスによる酸・塩基の定義}
  水溶液中で水素イオン\ce{H+}を放出する物質を酸,\ce{OH-}を放出する物質を塩基とする.
\end{example}

\begin{example}{「民主主義」の定義}{「民主主義」の定義}
  民主主義とは,国民が政治の意思決定に直接または間接的に参加する政治体制のことをいう.
\end{example}

どちらもわかりやすい例であり,「すでに知っていること」と思われる読者もいるかもしれない.
しかし,議論を行う上で,定義が曖昧なままだと不都合が生じることがある.

\begin{example}{曖昧な定義による誤解}{曖昧な定義による誤解}
  ある物体の速度$v$ \footnote{$\vec{v}$や$\bm{v}$と表記することもあるが,大学では,文字を書く際にベクトルとスカラーの表記を区別しない場合が多いので,ここもそれにならう.}は
  \[
    v = \frac{dx}{dt}
  \]
  と定義されるが,ここではこれを「速さ」と混同して使うとする.

  物理学では「速度」は大きさと方向を持つベクトル量であり,「速さ」はその大きさだけを示すスカラー量である.
  実際,この場合の「速さ」を数式で表すと,$v$ではなく$\norm{v}$となる.

  このような違いを明確にしないと,運動の解析において誤った結論を導く可能性がある.
\end{example}

同じ事柄について,2つ以上の定義の形式があることもある.次は数学における例を見てみよう:

\phantomsection
\begin{example}{「絶対値」の定義}{「絶対値」の定義}
  $ x \in \mathbb{R}$に対して,$x$の絶対値を$\abs{x}$とかき,次のように定義する:
  \[
    \abs{x} \coloneqq \max \{ x , -x \} .
  \]
  この定義は以下のような形式で表現してもよい:
  \[
    \abs{x} \coloneqq  \sqrt{x^2}.
  \]
\end{example}

次に,定義の性質についてみていこう.
線型代数の講義で学ぶことであるが,逆行列の定義は以下のようになる:

\phantomsection
\begin{example}{「逆行列」の定義}{「逆行列」の定義}
  正方行列$A$に対して,$BA = AB =E$となるような$B$が存在するとき,このような$B$を$A$の逆行列という.
\end{example}


ここで注意するのは \emph{数学では定義は最小限の情報にとどめることが慣習となっている}ということである.たとえば,\exref{「逆行列」の定義}から,以下の命題\footnote{「命題」については,のちほど詳しくみていくとする.ここでは「証明を与えるべきである主張」という理解でよい.}が成り立ち,その証明は容易である.

\phantomsection
\begin{prop}{逆行列の一意性}{逆行列の一意性}
  正方行列$A$に対して,$A$の逆行列は存在するとすればただひとつである.
\end{prop}

\begin{tleftbar}
  \begin{proof}
    $A$の逆行列が$B$,$C$であるとすると,
    \[
      AB=BA=E,\quad AC=CA=E
    \]
    が成り立つ.このとき,
    \[
      B=BE=B(AC)=(BA)C=EC=C.
    \]
    よって,$B=C$であり,逆行列の一意性が示された.
  \end{proof}
\end{tleftbar}

このことから,\exref{「逆行列」の定義}でとりあげた逆行列の定義をもっと詳しく

\begin{quote}
  正方行列$A$に対して,$BA = AB =E$となるような$B$が存在するとき,このような$B$はただひとつで,$B$を$A$の逆行列という.
  $B$は一意に存在するので,これを$A^{-1}$と記す\footnote{一意に存在することがわかっていないと,$A^{-1}$のような記法で表すことはためらわれる.}.もちろん
  \[
    AA^{-1}=A^{-1}A=E.
  \]
\end{quote}
と,「逆行列の一意性を定義に含めてもいいのではないか.」と主張する人がいるかもしれない.
ただ,数学では「定義は最小限の情報にとどめ,そこから導かれる主張を命題として証明する」という慣習があり,
なにが「定義」で,なにが「証明すべきこと」であるかはっきりさせることが多い\footnote{ただ,実際はここで取り上げた逆行列の定義も情報過多である.線型代数で学ぶことになるが,$AB=E$と$BA=E$のどちらか片方の式のみで定義してよいからである.}.

\phantomsection
\subsubsection{命題}

\emph{命題}\index{めいだい@命題}(\emph{proposition})とは,真偽が定まっている文を指す\footnote{命題はふたつの意味があり,ここでいう命題は「定理・命題・補題」などと並列して表記される「狭い意味での命題」というよりかは「真偽が決まっている文」というニュアンスを持つ「広い意味での命題」である.「狭い意味での命題」は正しい主張である.
  たとえば,「$3$以上の自然数$n$に対して$x^n + y^n =z^n$は自然数解を持たない」という「フェルマーの最終定理」は,
  アンドリュー・ワイルズが証明するまでは「真偽が決まっているがどちらかはわからない」という「広い意味での命題」であったが,
  証明されたのちに「広い意味での命題」であると同時に「狭い意味での命題」にもなった.}.\textbf{Prop}と略記される.
このような論理体系を\emph{二値論理}\index{にちろんり@二値論理}という.
二値論理においては,命題は「真」か「偽」のどちらかあり,命題が真であることを$\mathrm{T}$,偽であることを$\mathrm{F}$と表す.

中間として扱われる主張には
\begin{enumerate}[(1)]
  \item 定義が曖昧なもの \label{enu:定義が曖昧なもの}
  \item 意味が曖昧なもの \label{enu:意味が曖昧なもの}
  \item パラドックス \label{enu:パラドックス}
\end{enumerate}
などがある.\ref{enu:定義が曖昧なもの},\ref{enu:意味が曖昧なもの}についてはのちほど説明するとして,ここでは\ref{enu:パラドックス}について例をあげよう.

\phantomsection
\begin{example}{自己言及のパラドックス}{自己言及のパラドックス}
  次のような主張を考える:
  \begin{quote}
    この文は偽である.
  \end{quote}
  この主張は,自己言及のパラドックスであり,真偽が決まらない.

  以下も自己言及のパラドックスの例である:
  \begin{quote}
    「この壁に貼り紙をしてはならない」と書かれた貼り紙
  \end{quote}
\end{example}

ここまでで,二値論理でとりあげない主張を述べてきたが,そろそろ二値論理で取り扱う主張のお話に戻ろう.
また,以下では簡単のために,「広い意味での命題」と「狭い意味での命題」のどちらも「命題」と記すと約束する.

\phantomsection
\begin{example}{命題}{命題}
  次に示す文は真偽が真の命題である:
  \begin{enumerate}[(1)]
    \item 「1+1=2」
    \item 「 23 > 17」
    \item 「円周率は100未満である」
    \item 「霞ヶ浦は日本で二番目に面積が大きい湖である」
  \end{enumerate}
  また,次に示す文は真偽が偽の命題である:
  \begin{enumerate}[(a)]
    \item 「1+1=46」
    \item 「23>26」
    \item 「円周率は3未満である」
    \item 「霞ヶ浦は日本で五番目に面積が大きい湖である」
  \end{enumerate}
\end{example}

\phantomsection
\begin{example}{命題でない文}{命題でない文}
  次に示す文は命題ではない:
  \begin{enumerate}[(A)]
    \item 「$1+1$」(なにも主張しておらず,真か偽か判定できない)
    \item 「霞ヶ浦の面積は大きい」(客観的に大きいか判定できない)
    \item 「桃はおいしい」(基準が明確でなく,客観的に真か偽か判定できない)
    \item 「$x^2 > 4$」($x$に具体的な値を代入しないと真か偽か判定できない)\label{enu:命題でない文4}
  \end{enumerate}
\end{example}

\phantomsection
\subsubsection{定理}

\emph{定理}\index{ていり@定理}(\emph{theorem})とは,正しいと分かっている数学の主張\footnote{つまり,「狭い意味での命題」のこと.}の中でもとりわけ重要なものを指す.\textbf{Thm}と略記される.


定理は数学における真理であり,新たな理論の構築や他の結果の証明の基礎となる.定理は厳密な論理的推論と証明によって裏付けられており,その証明は既知の定理,定義,公理,または論理的推論規則に基づいて行われる.

定理が重要な主張であるがゆえに,「余弦定理」・「加法定理」・「ハイネ・ボレルの被覆定理」など,固有の名前が与えられているものも存在する.
その固有の名前は,定理の主張の詳細,もしくは発見者の名前ににちなんで付けられることが多い.例えば,「三平方の定理」や「コーシー・シュワルツの不等式」である.

\phantomsection
\begin{example}{三平方の定理}{三平方の定理}
  三平方の定理の主張は以下のようになる:
  \begin{quotation}
    直角三角形の斜辺の長さを$c$とし,他の二辺の長さを$a$,$b$としたとき,
    \[
      a^2+b^2=c^2
    \]
    が成り立つ.
  \end{quotation}
\end{example}

\phantomsection
\begin{example}{コーシー・シュワルツの不等式}{コーシー・シュワルツの不等式}
  コーシー・シュワルツの不等式は,内積空間における基本的な不等式であり,以下のように表される:
  \begin{quotation}
    任意の内積空間において,ベクトル$u$と$v$に対して,
    \[
      \norm{\langle u, v \rangle} \leq \norm{u} \cdot \norm{v}
    \]
    が成り立つ.
  \end{quotation}
  この不等式は解析学や線型代数学など,多くの分野で重要な役割を果たす.
\end{example}

\phantomsection
\begin{example}{微分積分学の基本定理}{微分積分学の基本定理}
  この定理は微分と積分の関係を明らかにするものであり,以下のように述べられる:
  \begin{quotation}
    $f$を区間$[a, b]$上で連続な実数値関数とする.このとき,
    \[
      F(x) = \int_a^x f(t) \, dt
    \]
    と定義すると,$F$は$[a, b]$上で微分可能であり,
    \[
      F'(x) = f(x)
    \]
    が成り立つ.
  \end{quotation}
  この定理により,積分と微分が互いに逆操作であることが示される.
\end{example}
\phantomsection
\subsubsection{補題・系}

\emph{補題}\index{ほだい@補題}(\emph{lemma})とは,定理や命題を証明する際に,その証明の一部として利用される補助的な命題のことを指す.\textbf{Lem}と略記される.補題は,直接的に重要な結果でない場合もあるが,より複雑な定理を証明するための重要なステップとなる主張である.

一方,\emph{系}\index{けい@系}(\emph{corollary})とは,既に証明された定理や命題から直接的に導かれる結果のことである.\textbf{Cor}と略記される.系は先の結果を応用することで容易に得られる新たな命題であり,元の定理の応用例や特別な場合を示すことが多い.

補題や系を説明するためには,関連するいくつかの命題や定理が必要である.以下に具体的な例を示す.

\phantomsection
\begin{example}{(補題)ユークリッドの補題}{(補題)ユークリッドの補題}
  ユークリッドの補題は,整数論における基本的な結果であり,素因数分解の一意性を証明する際に重要な役割を果たす.その主張は以下の通りである:

  \begin{quote}
    素数$p$が整数$a$と$b$の積$ab$を割り切るならば,$p$は$a$と$b$の少なくとも一方を割り切る.
  \end{quote}

\end{example}

\begin{tleftbar}
  \begin{proof}
    素数$p$が$ab$を割り切るとする.もし$p$が$a$を割り切らないならば,$\gcd(p, a) = 1$である.
    このとき,ベズーの等式より,ある整数$s$と$t$が存在して,$sp + ta = 1$が成り立つ.
    両辺に$b$を掛けると,$spb + tab = b$となる.
    左辺の$spb$は$p$で割り切れるが,$tab$は$p$で割り切れるため,
    右辺の$b$も$p$で割り切れる.したがって,$p$は$b$を割り切る.
  \end{proof}
\end{tleftbar}

この補題を用いて,素因数分解の一意性(算術の基本定理)を証明することができる.

\phantomsection
\begin{example}{自然数の約数の個数}{自然数の約数の個数}
  自然数$n$の正の約数の個数$d(n)$は,$n$の素因数分解に基づいて以下の式で与えられる:

  \begin{equation}
    d(n) = (e_1 + 1)(e_2 + 1) \dotsm (e_k + 1)
  \end{equation}

  ただし,$n$を素因数分解して$n = p_1^{e_1} p_2^{e_2} \dotsm p_k^{e_k}$と表した.
\end{example}

\begin{tleftbar}
  \begin{proof}
    各素因数$p_i$について,指数$e_i$は$0$から$e_i$までの値を取り得る.
    したがって,各$p_i$に対する約数の取り得る指数の個数は$e_i + 1$個である.
    全ての素因数について独立に指数を選ぶことで,
    $n$の全ての約数を生成できるため,約数の総数は各$(e_i + 1)$の積となる.
  \end{proof}
\end{tleftbar}

この命題は,素因数分解の一意性から直接的に導かれる結果である.

そしてこの例から,次の系が導かれる:

\phantomsection
\begin{example}{(系)約数の個数が奇数となる条件}{(系)約数の個数が奇数となる条件}
  自然数$n$の約数の個数$d(n)$が奇数となるための必要十分条件は,$n$が完全平方数であることである.
\end{example}

\begin{tleftbar}
  \begin{proof}
    約数の個数$d(n) = (e_1 + 1)(e_2 + 1) \dotsm (e_k + 1)$である.
    各$(e_i + 1)$が奇数となるためには,$e_i$が偶数でなければならない.
    つまり,全ての素因数の指数$e_i$が偶数であるとき,$n$は各素因数の偶数乗の積であり,
    これは$n$が完全平方数であることと同値である.
    逆に,$n$が完全平方数であれば,各$e_i$は偶数であり,したがって$d(n)$は奇数となる.
  \end{proof}
\end{tleftbar}

\phantomsection
\subsection{言い回し}
\phantomsection
\subsubsection{存在}



数学の証明において,「存在\index{そんざい@存在}」は重要な概念である.といっても,あまりこのことを意識したことのない読者の方もいると思うので,この場を借りて具体例をもとに説明を試みることとする.

まず,「最大値・最小値の定理」を考えてみる.この定理は,
\begin{quotation}
  $[a,b]$で連続な関数$f$に対して,$f$は$[a,b]$上で最大値と最小値を持つ.
\end{quotation}
というものがある.

「最大値・最小値が存在するなんて当たり前だ」と思われる読者もいるかもしれない.しかし,本当にそれは自明なのか.
実際には,関数の連続性や区間の閉有界性といった条件が揃って初めて,最大値や最小値の「存在」を保証できる.この定理を証明するにあたっては,厳密な数学的議論が必要となる.

次に,「極限値の存在」を考えてみよう.

\begin{prop}{}{}
  以下のような漸化式で定められた数列$(x_n)_{n \in \mathbb{N}}$を考える:
  \[
    \begin{cases}
      x_1 =0 , \\
      x_{n+1}= \sqrt{x_n+2}.
    \end{cases}
  \]
  このとき,$(x_n)_{n \in \mathbb{N}}$は収束し,
  \[
    \lim_{n \to \infty} x_n =2
  \]
  である.
\end{prop}

\begin{tleftbar}
  \begin{proof}
    いくつかの補題を確認しつつ示す.
    \begin{lemma}{}{}
      任意の$n \in \mathbb{N}$に対して,
      \[
        x_n \leqq x_{n+1} < 2
      \]
      である.
    \end{lemma}

    \begin{dotleftbar}
      \begin{proof}
        数学的帰納法により,$x_n \leqq x_{n+1} < 2$を示す.
        \begin{enumerate}[(I)]
          \item \mbox{} \\ \label{proof:存在1}
                $n=1$のとき,$ x_1 =0$,$x_2= \sqrt{0+2}=\sqrt{2}$なので,$ 0 \leqq \sqrt{2} < 2$により,
                $x_1 \leqq x_2 < 2$である.
          \item \mbox{} \\ \label{proof:存在2}
                $k \in \mathbb{N}$を任意にとり,$x_k \leqq x_{k+1} < 2$であると仮定する.

                このとき,$x_{k+1} < 2$により$ x_{k+1}+2 < 4$なので,
                \[
                  x_{k+2}=\sqrt{x_{k+1} + 2} < 2
                \]
                である.

                また,$x_k \leqq x_{k+1}$により,$x_k +2  \leqq x_{k+1}+2$であるから,
                \[
                  x_{k+1}= \sqrt{x_k +2} \leqq \sqrt{x_{k+1}+2}=x_{k+2}.
                \]
                よって,$n=k+1$の場合にも成り立つ.
        \end{enumerate}
        以上\ref{proof:存在1},\ref{proof:存在2}により,任意の$n \in \mathbb{N}$について
        \[
          x_n \leqq x_{n+1} < 2
        \]
        である.
      \end{proof}
    \end{dotleftbar}

    \begin{lemma}{}{}
      $(x_n)_{n \in \mathbb{N}}$は$ 0<  x \leqq 2$なる極限$x$に収束する.
    \end{lemma}

    \begin{dotleftbar}
      \begin{proof}
        先の補題により,任意の$n \in \mathbb{N}$について,$ x_n \leqq x_{n+1} <2$なので,
        数列$(x_n)_{n \in \mathbb{N}}$は単調に増加し,上に有界である.

        よって
        \[
          x \coloneqq \sup \{ x_n \mid n \in \mathbb{N} \}
        \]
        が存在し,$x \leqq  2$である.

        $ x>0$は明らかであるから,以上の考察によりこの命題が証明された.
      \end{proof}
    \end{dotleftbar}

    ここまでで,$(x_n)_{n \in \mathbb{N}}$の極限値の存在が確認されたので,先の漸化式について,
    \[
      x=\sqrt{x+2}
    \]
    とする.これを解くと$ x= -1,2$であるが,$x>0$により$x=2$である.

    以上の考察により,数列$(x_n)_{n \in \mathbb{N}}$は収束し,
    \[
      \lim_{n \to \infty} x_n =2.
    \]
  \end{proof}
\end{tleftbar}

ここまで考察して,ようやく数列$(x_n)_{n \in \mathbb{N}}$が収束することを確認できた.
この例では「極限値の存在」を確認する意味がわからない読者もいるかもしれないので,もう一つ例を挙げよう.


以下のような漸化式で定められた数列$(y_n)_{n \in \mathbb{N}}$を考える:
\[
  \begin{cases}
    y_1 =1 , \\
    y_{n+1}= 2y_n +2.
  \end{cases}
\]
この数列の極限値が存在すると仮定して,それを$y$とおこう.
\begin{tleftbar}
  与えられた漸化式により,
  \begin{align*}
     & y = 2y +2,           \\
     & \therefore ~ y = -2.
  \end{align*}
  しかし,
  \[
    \lim_{n \to \infty} y_n = \infty
  \]
  なので,$y=-2$は誤りである.
\end{tleftbar}

このように,数列の極限が存在することを確認しないと,誤った結論に到達することがある.この例から分かるように,存在を確認することは重要なことであるのだ.


\phantomsection
\subsubsection{一意性}

数学において「一意性\index{いちいせい@一意性}」が重要である場面は多い.読者の中には線型代数の講義で「逆行列の一意性」などに触れた方もいると思われる.なぜ「一意性」が重要であるのか,一つ例を挙げて考えてみる.

微分積分の講義で習う定理に「平均値の定理」というものがある.
その主張は
\begin{quotation}
  $[a,b]$で連続,$(a,b)$で微分可能な関数$f$に対して,
  \[
    \frac{f(b)-f(a)}{b-a}=f'(c)
  \]
  をみたす$ c \in (a,b)$が存在する.
\end{quotation}

というものである.証明はのちに述べるとして,この定理の主張を少し変更してみよう:

\begin{quotation}
  $[a,b]$で連続,$(a,b)$で微分可能な関数$f$に対して,
  \[
    \frac{f(b)-f(a)}{b-a}=f'(c)
  \]
  をみたす$ c \in (a,b)$がただひとつ存在する.
\end{quotation}
\begin{wrapfigure}[10]{r}{0.425\textwidth}
  %\centering
  \begin{tikzpicture}[scale=0.78]
    \begin{axis}[
        axis lines=middle,
        xmin=-1.2, xmax=1.2,
        ymin=-1.5, ymax=1.5,
        xtick=\empty, ytick=\empty, % 目盛を消す
        xlabel={}, ylabel={},       % ラベルを消す
        samples=200,
        domain=-1.2:1.2,
        width=10cm,
        height=8cm,
      ]
      % 関数 f(x) = x^3 のグラフ
      \addplot [dblue, thick] {x^3};

      % 点 (-1, -1) と (1, 1) を結ぶ割線
      \addplot [dred, thick] coordinates {(-1, -1) (1, 1)};

      % 平均値の定理を満たす点 x = -1/√3 の接線
      \def\xone{-1/sqrt(3)}
      \def\yone{(\xone)^3}
      \addplot [dgreen, dashed] {1*(x - \xone) + \yone};

      % 平均値の定理を満たす点 x = 1/√3 の接線
      \def\xtwo{1/sqrt(3)}
      \def\ytwo{(\xtwo)^3}
      \addplot [dgreen, dashed] {1*(x - \xtwo) + \ytwo};

      % 重要な点をマーク
      \addplot [only marks, mark=*, mark options={fill=black}] coordinates {(\xone, \yone) (\xtwo, \ytwo)};
      \addplot [only marks, mark=*, mark options={fill=dred}] coordinates {(-1, -1) (1, 1)}; % 赤い丸に変更
    \end{axis}
  \end{tikzpicture}
\end{wrapfigure}

ここで「$c \in (a,b)$が存在する」という主張を「$c \in (a,b)$がただひとつ存在する」というより強い主張に変更した.この主張の真偽は偽である.このことが問題になる状況を挙げよう.

$f(x)=x^3$という関数を考える.この関数は$[-1,1]$で連続,$(-1,1)$で微分可能である.
このとき,図のように,条件を満たす$c \in (-1,1)$は複数存在するので,「ただひとつ存在する」という主張は偽である.
さらに言えば,2本より多くこのような接線を引ける場合もある\footnote{確認してみよ.}.

このことから,安易に「ただ一つ存在する」などと強い主張をすることは避けるべきであることがわかる.このことは平均値の定理に限らず,中間値の定理なども同様である.


\phantomsection
\subsubsection{かつ・または}

数学において,「かつ」は日常とほとんど同じ意味で使われるが,「または」の使い方は日常とは異なる.以下に例を挙げよう.

\phantomsection
\begin{example}{「または」の使用例}{「または」の使用例}
  \begin{enumerate}[(A)]
    \item ランチメニューの主食として,米またはパンがついてくる\footnote{この文を「米とパンの両方が食べられる」と解釈してもらっては困る.}.\label{enu:米またはパン}
    \item 「運転免許を持っていない人」または「18歳未満の人」はレンタカーを借りることができない. \label{enu:運転免許または18歳未満}
  \end{enumerate}
  \ref{enu:米またはパン}は「どちらか片方のみ」の意味で「または」を使い,\ref{enu:運転免許または18歳未満}は「いずれかが」の意味で「または」を用いている.
\end{example}

数学では,「または」は「いずれかが」の意味で使われ,「どちらか片方のみ」の意味で使われることはない.

たとえば,$A$,$B$を集合とするとき,
\[
  x \in A \cup B
\]
は「$x$は$A$の元であるか,$B$の元であるか,あるいは両方である」という意味である.
つまり,「$ x \in A$であり,$x \notin  B$である」あるいは「$ x \notin A$であり,$x \in B$である」といった状況のときにも,$x \in A \cup B$と記す.
%ここまで

\subsubsection{任意の/すべての}

数学において,「任意の\index{にんいの@任意の}」と「すべての\index{すべての@すべての}」はニュアンスが異なる.
英語では,「任意の」は``for any'',「すべての」は``for all''に相当する.
``for any''のニュアンスは「複数の対象があり,その中のどの一つをとっても」という意味であり,``for all''は「対象全体を見て,すべてのものが」という意味である.

このことからわかるように,「任意の」を「すべての」で置き換える場合には,このようなニュアンスの違いに注意しなければならない.

\subsubsection{簡単のため\index{かんたんのため@簡単のため}}

「簡単にするために」のほうがしっくりくる方もおられるかもしれないが,これは``For simplicity''の訳で,
その後の議論を簡単にするために「議論することが楽な仮定」を設ける際に使う言葉である.

\begin{example}{「簡単のため」の使用例}{「簡単のため」の使用例}
  \begin{enumerate}[(A)]
    \item 簡単のため,平行移動して頂点が原点にある形として,$f(x)=ax^2$を考える.
  \end{enumerate}
\end{example}

\subsubsection{従う\index{したがう@従う}}

数学において「従う」は日常とはまた違った意味で使われる.日常だと「王に従う」や「上司に従う」など,「命令を受けてそれに従う」という意味で使われることが多いが,
数学では「$A$という事柄から,すぐ$B$という事柄がわかる」という場合に「$A$から$B$が従う」という.

\begin{example}{「従う」の使用例}{「従う」の使用例}
  \begin{enumerate}[(A)]
    \item $n$が偶数であることから,$n^2$が偶数であることが従う.\label{enu:偶数ならば偶数}
    \item $G$が群であることから,$G \ne \varnothing$が従う\footnote{群であれば単位元が存在することは群の定義からわかる.そのため群は空集合でない.}.\label{enu:群の例}
  \end{enumerate}
\end{example}

\subsubsection{嬉しい\index{うれしい@嬉しい}}

日常では,「嬉しい」という言葉は「喜びを感じる」という意味で使われることが多いが,数学の文脈では「都合がよい」,「議論が楽になる」という意味でたびたび使われる.

\begin{example}{「嬉しい」の使用例}{「嬉しい」の使用例}
  \[
    \int_{\alpha}^{\beta} (x-\alpha)(x-\beta)\, dx = -\frac{(\beta-\alpha)^3}{6}
  \]
  であることを示したい.このときに
  \[
    \int_{\alpha}^{\beta} (x-\alpha)(x-\beta)\, dx = \int_{\alpha}^{\beta}  (x-\alpha)\{(x-\alpha) - (\beta-\alpha)\} \, dx
  \]
  と変形してなにが嬉しいかというと,
  \begin{align*}
    \int_{\alpha}^{\beta}  (x-\alpha)\{(x-\alpha) - (\beta-\alpha)\} \, dx & = \int_{\alpha}^{\beta} \{  (x-\alpha)^2 -(\beta-\alpha)(x-\alpha) \} \, dx                        \\
                                                                           & = \Biggl [ \frac{(x-\alpha)^3}{3} - \frac{(\beta-\alpha)(x-\alpha)^2}{2} \Biggr ]_{\alpha}^{\beta} \\
                                                                           & = - \frac{1}{6} (\beta-\alpha)^3
  \end{align*}
  というふうに,簡単に計算できるからである.
\end{example}



\subsubsection{おさえる\index{おさえる}・評価\index{ひょうか@評価}する}

数学の文脈では「おさえる」という言葉は,ある数$M$を,別の数 $K$を用いて,$K<M$という形で表すことを指す.
より詳しく「上からおさえる」と表すこともある.

また,数学の文脈で「評価」という言葉は,ある数$M$を不等式で表し,その数がどの程度の大きさであるかを示すことを指す.
たとえば,$ 1 < M <4$と表すことは「$M$を$1$より大きく$4$より小さいと評価する」ということを指す.
さらに,上記の$M$を$ 2< M <3$と表すことができる場合は,後者の表現のほうが「評価が厳しい」ということができる.

\begin{example}{「おさえる」と「評価する」の使用例}{「おさえる」と「評価する」の使用例}
  \begin{enumerate}[(A)]
    \item 数列$(a_n)_{n \in \mathbb{N}}$を $ a_n <M$と上からおさえる.
    \item $e$の値を$2.7 < e < 2.8$と評価する.
  \end{enumerate}
\end{example}


\subsubsection{特徴づけ\index{とくちょうづけ@特徴づけ}}

数学の文脈では,「特徴づけ」という言葉がよく使われる.これは,ある対象が持つ性質を使ってその対象をただひとつと特定することを指す.

\begin{example}{「特徴づけ」の使用例}{「特徴づけ」の使用例}
  \begin{enumerate}[(A)]
    \item 実数の連続性の公理は,実数の集合を特徴づけるものである\footnote{この言葉の意味がよくわからない読者は,解析学の教科書を参照してみることを推奨する.}.
  \end{enumerate}
\end{example}
